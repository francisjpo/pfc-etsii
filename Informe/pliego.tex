%!TEX root = informe.tex

\setcounter{chapter}{0}
\chapter{Condiciones técnicas}
% \addcontentsline{toc}{chapter}{Condiciones técnicas}

\section{Generalidades}

Los principios del Análisis de Ciclo de Vida son fundamentales y se deberán utilizar como orientación para tomar decisiones relacionadas tanto con la planificación como con la realización del análisis.

\section{Apreciación general del ciclo de vida}

El Análisis de Ciclo de Vida considerará el ciclo de vida completo del producto, desde la extracción y adquisición de la materia prima, pasando por la producción de energía y materia, y la fabricación, hasta el uso y el tratamiento al final de la vida útil y la disposición final. A través de esta visión general, se identificará y intentará evitar el desplazamiento de una carga ambiental potencial entre las etapas del ciclo de vida o los procesos individuales.

\section{Enfoque ambiental}
El Análisis de Ciclo de Vida tratará los aspectos e impactos ambientales del sistema del producto. Los aspectos e impactos económicos y sociales estarán fuera del alcance del Análisis de Ciclo de Vida. Se podrán combinar otras herramientas con el Análisis de Ciclo de Vida para análisis más profundos.

\section{Enfoque relativo y Unidad Funcional}
El Análisis de Ciclo de Vida será un enfoque relativo, que se estructurará alrededor de una Unidad Funcional. Esta Unidad Funcional definirá el estudio. Todos los análisis subsecuentes serán, por tanto, relativos a esa Unidad Funcional, ya que todas las entradas y salidas en el Inventario de Ciclo de Vida (ICV), y consecuentemente el perfil de la Evaluación del Impacto del Ciclo de Vida (EICV), se relacionarán con la Unidad Funcional.

\section{Enfoque iterativo}
El Análisis de Ciclo de Vida será una técnica iterativa. Las fases individuales del Análisis de Ciclo de Vida utilizarán resultados de las otras fases. El enfoque iterativo en y entre las fases contribuirá a la integridad y coherencia del estudio y de los resultados presentados.

\section{Transparencia}
Debido a la complejidad inherente al Análisis de Ciclo de Vida, la transparencia será un principio guía importante en la realización del Análisis de Ciclo de Vida, a fin de asegurar una adecuada interpretación de los resultados.

\section{Integridad}
El Análisis de Ciclo de Vida considerará todos los atributos o aspectos del entorno natural, de la salud humana y de los recursos. La consideración en un único estudio y con una perspectiva transversal de todos los atributos y aspectos, se podrán identificar y evaluar las compensaciones potenciales.

\section{Prioridad del enfoque científico}
Las decisiones en el Análisis de Ciclo de Vida se basarán preferentemente en las ciencias naturales. Si esto no es posible, se podrán utilizar otros enfoques, como las ciencias económicas y sociales, o se puede hacer referencia a convencio- nes internacionales. Si no existiera una base científica ni una justificación basada en otros enfoques o en convenciones internacionales, las decisiones se podrán basar en juicios de valor.

\section{Alcance}
Cuando se defina el alcance del Análisis de Ciclo de Vida, se considerará el contexto de la toma de decisión; es decir, los sistemas del producto estudiados deberán tratar adecuadamente los productos y procesos afectados por la aplicación prevista.

Los ejemplos de aplicación se referirán a decisiones que pretendan conseguir mejoras ambientales, lo que también constituye el enfoque global de la serie ISO 14000. Por lo tanto, los productos y procesos estudiados en un Análisis de Ciclo de Vida son aquellos afectados por la decisión que el Análisis de Ciclo de Vida pretende apoyar.

\chapter{Condiciones administrativas y legales}
% \addcontentsline{toc}{chapter}{Condiciones administrativas y legales}

\section{Autoría}
El autor de este proyecto cede al 50\% los derechos derivados de este proyecto al Departamento de Expresión Gráfica, Diseño y Proyectos de la Escuela Técnica Superior de Ingeniería Industrial de la Universidad de Málaga.

\section{Realización y supervisión}
El presente proyecto será realizado por el autor del mismo, bajo dirección y supervisión del tutor. Si esto no fuera posible, dicha realización y asesoría debería ser llevada a cabo por personal del Departamento de Expresión Gráfica, Diseño y Proyectos de la Escuela Técnica Superior de Ingeniería Industrial de la Universidad de Málaga.

\section{Cambios y desarrollos posteriores}
El autor del presente proyecto deberá ser puntualmente informado de los posibles cambios o modificaciones que pudiesen realizarse en el mismo.

En el caso de cambios o desarrollos posteriores de este proyecto se informará al autor para colaborar en el estudio o investigación que se este realizando.

\section{Consultas}
Se autoriza la consulta de este proyecto a toda persona autorizada por parte del Departamento de Expresión Gráfica, Diseño y Proyectos y a cualquier persona matriculada en la Universidad de Málaga que podrá solicitar el Proyecto en la Biblioteca de la Escuela Técnica Superior de Ingeniería Industrial de la Universidad de Málaga.

\vspace{1cm}
\today \hfill Fdo. Francisco José Pinto Oliver
