%!TEX root = informe.tex

\setcounter{chapter}{0}
\chapter{Condiciones técnicas}
% \addcontentsline{toc}{chapter}{Condiciones técnicas}

\section{Principios del ACV}
\subsection{Generalidades}

Los principios del Análisis de Ciclo de Vida son fundamentales y se deberán utilizar como orientación para tomar decisiones relacionadas tanto con la planificación como con la realización del análisis.

\subsection{Apreciación general del ciclo de vida}

El Análisis de Ciclo de Vida considerará el ciclo de vida completo del producto, desde la extracción y adquisición de la materia prima, pasando por la producción de energía y materia, y la fabricación, hasta el uso y el tratamiento al final de la vida útil y la disposición final. A través de esta visión general, se identificará y intentará evitar el desplazamiento de una carga ambiental potencial entre las etapas del ciclo de vida o los procesos individuales.

\subsection{Enfoque ambiental}
El Análisis de Ciclo de Vida tratará los aspectos e impactos ambientales del sistema del producto. Los aspectos e impactos económicos y sociales estarán fuera del alcance del Análisis de Ciclo de Vida. Se podrán combinar otras herramientas con el Análisis de Ciclo de Vida para análisis más profundos.

\subsection{Enfoque relativo y Unidad Funcional}
El Análisis de Ciclo de Vida será un enfoque relativo, que se estructurará alrededor de una Unidad Funcional. Esta Unidad Funcional definirá el estudio. Todos los análisis subsecuentes serán, por tanto, relativos a esa Unidad Funcional, ya que todas las entradas y salidas en el Inventario de Ciclo de Vida (ICV), y consecuentemente el perfil de la Evaluación del Impacto del Ciclo de Vida (EICV), se relacionarán con la Unidad Funcional.

\subsection{Enfoque iterativo}
El Análisis de Ciclo de Vida será una técnica iterativa. Las fases individuales del Análisis de Ciclo de Vida utilizarán resultados de las otras fases. El enfoque iterativo en y entre las fases contribuirá a la integridad y coherencia del estudio y de los resultados presentados.

\subsection{Transparencia}
Debido a la complejidad inherente al Análisis de Ciclo de Vida, la transparencia será un principio guía importante en la realización del Análisis de Ciclo de Vida, a fin de asegurar una adecuada interpretación de los resultados.

\subsection{Integridad}
El Análisis de Ciclo de Vida considerará todos los atributos o aspectos del entorno natural, de la salud humana y de los recursos. La consideración en un único estudio y con una perspectiva transversal de todos los atributos y aspectos, se podrán identificar y evaluar las compensaciones potenciales.

\subsection{Prioridad del enfoque científico}
Las decisiones en el Análisis de Ciclo de Vida se basarán preferentemente en las ciencias naturales. Si esto no es posible, se podrán utilizar otros enfoques, como las ciencias económicas y sociales, o se puede hacer referencia a convencio- nes internacionales. Si no existiera una base científica ni una justificación basada en otros enfoques o en convenciones internacionales, las decisiones se podrán basar en juicios de valor.

\subsection{Alcance}
Cuando se defina el alcance del Análisis de Ciclo de Vida, se considerará el contexto de la toma de decisión; es decir, los sistemas del producto estudiados deberán tratar adecuadamente los productos y procesos afectados por la aplicación prevista.

Los ejemplos de aplicación se referirán a decisiones que pretendan conseguir mejoras ambientales, lo que también constituye el enfoque global de la serie ISO 14000. Por lo tanto, los productos y procesos estudiados en un Análisis de Ciclo de Vida son aquellos afectados por la decisión que el Análisis de Ciclo de Vida pretende apoyar.

\section{Fases del ACV}

El estudio de ACV se compondrá de cuatro fases. Éstas serán:
\begin{itemize}
  \item definición del objetivo y el alcance,
  \item análisis del inventario,
  \item evaluación del impacto, y
  \item interpretación.
\end{itemize}

\section{Características esenciales de un ACV}

Las características esenciales en la metodología del ACV serán:
\begin{itemize}
  \item el ACV evaluará, de forma sistemática los aspectos e impactos ambientales de los sistemas del producto, desde la adquisición de la materia prima hasta la disposición final, de acuerdo con el objetivo y el alcance establecidos;
  \item la naturaleza relativa de un ACV se deberá a las características de la unidad funcional dentro de la metodología;
  \item el nivel de detalle y la duración de un ACV podrán variar de manera considerable, dependiendo de la definición del objetivo y el alcance;
  \item se establecerán disposiciones, dependiendo de la aplicación prevista del ACV, para respetar la confidencialidad y la propiedad;
  \item la metodología del ACV estará abierta a la inclusión de nuevos hallazgos científicos y mejoras en el estado del arte de la técnica;
  \item se aplicarán requisitos específicos a los ACV que se pretendan utilizar en las aseveraciones comparativas que serán divulgadas al público;
  \item no habrá un método único para realizar un ACV. Se tendrá flexibilidad para implementar un ACV según esté establecido en la norma UNE-EN-ISO 14040:2006, de acuerdo con la aplicación prevista;
  \item el ACV será diferente de otras técnicas (tales como la evaluación del desempeño ambiental, la evaluación del impacto ambiental y la evaluación del riesgo) ya que será un enfoque relativo basado en una unidad funcional; sin embargo, el ACV podrá utilizar la información obtenida con estas otras técnicas;
  \item el ACV tratará los impactos ambientales potenciales; el ACV no hará predicción de impactos ambientales absolutos o precisos debido a:
  \begin{itemize}
    \item la expresión relativa de los impactos ambientales potenciales con relación a una unidad de referencia,
    \item la integración de los datos ambientales en el espacio y en el tiempo,
    \item la incertidumbre inherente al modelar los impactos ambientales, y
    \item al hecho de que algunos impactos ambientales posibles sean claramente impactos futuros;
  \end{itemize}
  \item la fase de EICV, de forma conjunta con otras fases del ACV, proporcionará una amplia perspectiva de los asuntos ambientales y de los recursos para uno o más sistemas del producto;
  \item la EICV asignará los resultados del ICV a categorías de impacto; para cada categoría de impacto, se seleccionará un indicador de categoría de impacto de ciclo de vida y se calculará el resultado del indicador de categoría (resultado de indicador), la recopilación de resultados del indicador (resultados de la EICV) o el perfil de la EICV proporcionará información sobre los asuntos ambientales asociados con las entradas y salidas del sistema del producto;
  \item no habrá base científica para reducir los resultados del ACV a un único número o a una puntuación global, ya que la ponderación requiere juicios de valor;
  \item la interpretación del ciclo de vida utilizará un procedimiento sistemático para identificar, calificar, verificar, evaluar y presentar las conclusiones basadas en los hallazgos de un ACV, a fin de cumplir con los requisitos de la aplicación como se describe en el objetivo y el alcance del estudio;
  \item la interpretación del ciclo de vida utilizará un procedimiento iterativo tanto en la fase de interpretación como en las otras fases de un ACV;
  \item la interpretación del ciclo de vida establecerá disposiciones para los vínculos entre un ACV y otras técnicas de gestión ambiental, enfatizando las fortalezas y las limitaciones de un ACV en relación con la definición de su objetivo y alcance.
\end{itemize}

\section{Conceptos generales del sistema del producto}

El ACV modelará el ciclo de vida de un producto como su sistema del producto, el cual desempeñará una o más de las funciones definidas.

La propiedad fundamental de un sistema del producto se caracterizará por su función y no se podrá definir solamente en términos de los productos finales.

Los sistemas del producto se subdividirán en un conjunto de procesos unitarios. Los procesos unitarios se vincularán entre sí mediante flujos de productos intermedios y/o de residuos para tratamiento, con otros sistemas de producto, mediante flujos de producto, y con el medio ambiente mediante flujos elementales.

La división de un sistema del producto en los procesos unitarios que lo componen facilitará la identificación de las entradas y salidas del sistema del producto. En muchos casos, algunas de las entradas se utilizarán como un componente del producto resultante. Mientras que otras (entradas auxiliares) se utilizarán dentro de un proceso unitario pero no formarán parte del producto resultante. Un proceso unitario también generará otras salidas (flujos elementales y/o productos) como resultado de sus actividades. El nivel de detalle del modelado que se requerirá para satisfacer el objetivo del estudio determina los límites de un proceso unitario.

Los flujos elementales incluirán la utilización de recursos y las emisiones al aire, y los vertidos al agua y al suelo asociados con el sistema. Podrán inferirse interpretaciones de estos datos, dependiendo del objetivo y el alcance del ACV. Estos datos serán el resultado del ICV y constituirán la entrada para la EICV.

\section{Metodología}
\subsection{Requisitos generales}

Al desarrollar un ACV, se deberán aplicar los requisitos de la norma UNE-EN-ISO 14044:2006.

\subsection{Definición del objetivo y del alcance}
El objetivo de un ACV establecerá:
\begin{itemize}
  \item la aplicación prevista,
  \item las razones para realizar el estudio,
  \item el público previsto, es decir las personas a quienes se prevé comunicar los resultados del estudio, y
  \item si se prevé utilizar los resultados en aseveraciones comparativas que se divulgarán al público.
\end{itemize}

El alcance deberá estar suficientemente bien definido para asegurar que la amplitud, profundidad y el nivel de detalle del estudio sean compatibles y suficientes para alcanzar el objetivo establecido, e incluirá los siguientes puntos:
\begin{itemize}
  \item el sistema del producto a estudiar;
  \item las funciones del sistema del producto o, en el caso de estudios comparativos, los sistemas;
  \item la unidad funcional;
  \item los límites del sistema;
  \item los procedimientos de asignación;
  \item las categorías de impacto seleccionadas y la metodología de evaluación de impacto, y la subsecuente interpretación a utilizar;
  \item requisitos relativos a los datos;
  \item las suposiciones;
  \item las limitaciones;
  \item los requisitos iniciales de calidad de los datos;
  \item el tipo de revisión crítica, si la hay;
  \item el tipo y formato del informe requerido para el estudio.
\end{itemize}

La técnica de ACV será iterativa, y mientras se recopilan los datos e información, podrán tener que modificarse diversos
aspectos del alcance para cumplir con el objetivo original del estudio.

\subsection{Función, unidad funcional y flujos de referencia}

El sistema podrá tener varias funciones posibles y la(s) seleccionada(s) para el estudio dependerá(n) del objetivo y alcance del ACV.

La unidad funcional definirá la cuantificación de las funciones identificadas (características de desempeño) del producto. El propósito fundamental de la unidad funcional será proporcionar una referencia a la cual se relacionarán las entradas y salidas. Se necesitará esta referencia para asegurar que los resultados del ACV son comparables. El carácter comparativo de los resultados de los ACV será particularmente crítico cuando se estén evaluando sistemas diferentes, dado que habrá que asegurar que estas comparaciones se hagan sobre una base común.

Será importante determinar el flujo de referencia en cada sistema del producto, para cumplir con la función prevista, es decir, la cantidad de productos necesaria para cumplir la función.

\subsection{Límites del sistema}

El ACV se realizará definiendo los sistemas de producto como modelos que describan los elementos clave de los sistemas físicos. Los límites del sistema definirán los procesos unitarios a ser incluidos en el sistema. Idealmente, el sistema del producto se deberá modelar de tal manera que las entradas y las salidas en sus límites sean flujos elementales. Sin embargo, no será necesario gastar recursos para cuantificar tales entradas y salidas que no producirán cambios significativos en las conclusiones generales del estudio.

La elección de los elementos del sistema físico a modelar dependerá de la definición del objetivo y el alcance del estudio, de su aplicación y público previstos, de las suposiciones realizadas, de las restricciones en cuanto a datos y costos y los criterios de corte. Los modelos utilizados deberán describirse y las suposiciones que fundamentan esas elecciones deberán identificarse. Los criterios de corte utilizados en un estudio deberán ser claramente entendidos y descritos.

Los criterios utilizados para establecer los límites del sistema serán importantes para el grado de confianza en los resultados de un estudio y la posibilidad de alcanzar su objetivo.

Cuando se establezcan los límites del sistema, se deberán considerar varias etapas del ciclo de vida, procesos unitarios y flujos, como por ejemplo:
\begin{itemize}
  \item adquisición de materias primas,
  \item entradas y salidas en la secuencia principal de fabricación/procesamiento;
  \item distribución/transporte;
  \item producción y utilización de combustibles, electricidad y calor;
  \item utilización y mantenimiento de productos;
  \item disposición de los residuos del proceso y de los productos;
  \item recuperación de productos utilizados (incluyendo reutilización, reciclado y recuperación de energía);
  \item producción de materiales secundarios;
  \item producción, mantenimiento y desmantelamiento de los equipos;
  \item operaciones adicionales, tales como iluminación y calefacción.
\end{itemize}

\subsection{Requisitos de calidad de los datos}

Los requisitos de calidad de los datos especificarán, en términos generales, las características de los datos necesarios para el estudio.

Las descripciones de la calidad de los datos serán importantes para comprender la fiabilidad de los resultados del estudio e interpretar correctamente los resultados del estudio.

\section{Análisis del inventario del ciclo de vida (ICV)}
\subsection{Generalidades}
El análisis del inventario implicará la recopilación de los datos y los procedimientos de cálculo para cuantificar las entradas y salidas pertinentes de un sistema del producto.

La realización del análisis de inventario es un proceso iterativo. A medida que se recopilen los datos y se aprenda más sobre el sistema, se podrán identificar nuevos requisitos o limitaciones, que requerirán cambios en los procedimientos de recopilación de datos, de manera que aún se puedan cumplir los objetivos del estudio. Algunas veces, se podrán identificar algunos asuntos que requieran una revisión del objetivo o del alcance del estudio.

\subsection{Recopilación de datos}

Los datos para cada proceso unitario dentro de los límites del sistema podrán clasificarse bajo grandes títulos que incluirán:
\begin{itemize}
  \item las entradas de energía, de materia prima, entradas auxiliares, otras entradas físicas,
  \item los productos, coproductos y residuos,
  \item las emisiones al aire, los vertidos al agua y suelo, y
  \item otros aspectos ambientales.
\end{itemize}

La recopilación de datos podrá ser un proceso intensivo en materia de recursos. Las limitaciones prácticas en la recopilación de datos deberán tenerse en cuenta en el alcance y documentarse en el informe del estudio.

\subsection{Cálculo de datos}
Después de la recopilación de datos, los procedimientos de cálculo, que incluirán:
\begin{itemize}
  \item la validación de los datos recopilados,
  \item la relación de los datos con los procesos unitarios, y
  \item la relación de los datos con el flujo de referencia de la unidad funcional,
\end{itemize}

serán necesarios para generar los resultados del inventario del sistema definido para cada proceso unitario y para la unidad funcional definida del sistema del producto que se va a modelar.

El cálculo de los flujos de energía deberá tener en cuenta las diferentes fuentes de combustibles y electricidad utilizadas, la eficiencia de la conversión y la distribución del flujo de energía, así como las entradas y salidas asociadas a la generación y a la utilización de ese flujo de energía.

\subsection{Asignación de flujos y de emisiones y vertidos}
Pocos procesos industriales producen una salida única o están basados en una relación lineal entre las entradas y las salidas de materias primas. De hecho, la mayoría de los procesos industriales producen más de un producto, y reciclan los productos intermedios o los residuos de productos.

Deberá considerarse la necesidad de procedimientos de asignación para los sistemas que incluyan productos múltiples y para los sistemas de reciclado.

\section{Evaluación del impacto del ciclo de vida (EICV)}
\subsection{Generalidades}
La fase de evaluación de impacto de un ACV tendrá como propósito evaluar cuán significativos son los impactos ambientales potenciales utilizando los resultados del ICV. En general, este proceso implicará la asociación de los datos de inventario con las categorías de impactos ambientales específicos y con los indicadores de esas categorías para entender estos impactos. La fase de la EICV también proporcionará información para la fase de interpretación del ciclo de vida.

La evaluación del impacto podrá incluir un proceso iterativo de revisión del objetivo y del alcance del estudio de ACV para determinar si se han cumplido los objetivos del mismo, o para modificar el objetivo y el alcance si la evaluación indica que no se pueden alcanzar.

Cuestiones tales como la elección, el modelado y la evaluación de categorías de impacto podrán introducir subjetividad en la fase de la EICV. Por lo tanto, la transparencia será crítica en la evaluación del impacto a fin de asegurar que las suposiciones están claramente descritas e informadas.


\subsection{Elementos de la EICV}
La separación de la fase de la EICV en elementos diferentes será útil y necesaria por varias razones:
\begin{itemize}
  \item cada elemento de la EICV se distinguirá y podrá definirse con claridad;
  \item la fase de definición del objetivo y del alcance de un ACV podrá considerar por separado cada elemento de la EICV;
  \item se podrá realizar para cada elemento una evaluación de la calidad de los métodos, suposiciones y otras decisiones de la EICV;
  \item será posible dar transparencia dentro de cada elemento de la EICV a los procedimientos, las suposiciones y otras operaciones para la revisión crítica y el informe;
  \item dentro de cada elemento, es posible dar transparencia a la utilización de valores y de la subjetividad (en adelante referido como juicios de valor), para la revisión crítica y el informe.
\end{itemize}

El nivel de detalle, la selección de impactos evaluados y las metodologías utilizadas dependerán del objetivo y del alcance del estudio.

\subsection{Limitaciones de la EICV}
La EICV tratará solamente los asuntos ambientales especificados en el objetivo y el alcance. Por lo tanto, la EICV no será una evaluación completa de todos los asuntos ambientales del sistema del producto bajo estudio.

La EICV no siempre podrá demostrar diferencias significativas entre las categorías de impacto y los resultados de sus indicadores correspondientes para diferentes alternativas de los sistemas del producto. Esto podrá deberse a:
\begin{itemize}
  \item un desarrollo limitado de los modelos de caracterización, de los análisis de sensibilidad y de incertidumbre para la fase de la EICV,
  \item limitaciones de la fase de ICV, tales como el establecimiento de los límites del sistema, que no incluyan todos los procesos unitarios posibles para un sistema del producto o no incluya todas las entradas y salidas de cada proceso unitario, ya que puede que haya cortes y vacíos en los datos,
  \item limitaciones de la fase de ICV, tales como una calidad inadecuada de los datos del ICV, que pueda originarse, por ejemplo, por las incertidumbres o las diferencias en los procedimientos de asignación y de agregación, y
  \item limitaciones en la recopilación de los datos de inventario adecuados y representativos para cada categoría de impacto.
\end{itemize}

La ausencia de dimensiones espaciales y temporales en los resultados del ICV introducirán incertidumbre en los resultados de la EICV. La incertidumbre varíará según las características espaciales y temporales de cada categoría de impacto.

No habrá metodologías aceptadas de manera general para asociar de forma coherente y exacta los datos de inventario con los impactos ambientales potenciales específicos. Los modelos de categorías de impacto se encontrarán en diferentes etapas de desarrollo.



\section{Interpretación del ciclo de vida}

La interpretación será la fase del ACV, en la cual los hallazgos del análisis del inventario y de la evaluación de impacto se considerarán juntos. La fase de interpretación deberá proporcionar resultados que sean coherentes con el objetivo y el alcance definidos, que lleguen a conclusiones, expliquen las limitaciones y proporcionen recomendaciones.

La interpretación deberá reflejar el hecho de que los resultados de la EICV estén basados en un enfoque relativo, indiquen efectos ambientales potenciales, no predigan los impactos reales en los puntos finales de categoría, ni si se sobrepasan los umbrales, los márgenes de seguridad ni los riesgos.

Los hallazgos de esta interpretación podrán dar como resultado conclusiones y recomendaciones para la toma de decisiones, coherentes con el objetivo y alcance del estudio.

La interpretación del ciclo de vida intentará ofrecer una lectura comprensible, completa y coherente de la presentación de resultados de un ACV, de acuerdo con la definición del objetivo y el alcance del estudio.

La fase de interpretación podrá involucrar un proceso iterativo de revisión y de actualización del alcance de un ACV, así como de la naturaleza y de la calidad de los datos recopilados de modo que sean coherentes con el objetivo definido.

Los hallazgos de la interpretación del ciclo de vida deberán reflejar los resultados del elemento de evaluación.

\chapter{Condiciones administrativas y legales}
% \addcontentsline{toc}{chapter}{Condiciones administrativas y legales}

\section{Autoría}
El autor de este proyecto cede al 50\% los derechos derivados de este proyecto al Departamento de Expresión Gráfica, Diseño y Proyectos de la Escuela Técnica Superior de Ingeniería Industrial de la Universidad de Málaga.

\section{Realización y supervisión}
El presente proyecto será realizado por el autor del mismo, bajo dirección y supervisión del tutor. Si esto no fuera posible, dicha realización y asesoría debería ser llevada a cabo por personal del Departamento de Expresión Gráfica, Diseño y Proyectos de la Escuela Técnica Superior de Ingeniería Industrial de la Universidad de Málaga.

\section{Cambios y desarrollos posteriores}
El autor del presente proyecto deberá ser puntualmente informado de los posibles cambios o modificaciones que pudiesen realizarse en el mismo.

En el caso de cambios o desarrollos posteriores de este proyecto se informará al autor para colaborar en el estudio o investigación que se este realizando.

\section{Consultas}
Se autoriza la consulta de este proyecto a toda persona autorizada por parte del Departamento de Expresión Gráfica, Diseño y Proyectos y a cualquier persona matriculada en la Universidad de Málaga que podrá solicitar el Proyecto en la Biblioteca de la Escuela Técnica Superior de Ingeniería Industrial de la Universidad de Málaga.

\vspace{1cm}
\today \hfill Fdo. Francisco José Pinto Oliver
