%!TEX root = informe.tex
\chapter{Análisis de Ciclo de Vida: fin de vida}
%Manual Euroadoquín2
\section{Fin de vida}

La finalidad del reciclaje es conseguir un objetivo de cero residuos utilizando todos los materiales de subproductos encontrados en la rehabilitación o reconstrucción del pavimento. Esto no es sólo una ventaja económica, sino que el reciclaje local minimiza el impacto medioambiental reduciendo la huella de carbono, la energía utilizada y las emisiones, además de reducir la necesidad de vertederos y la extracción de materias primas no renovables.

El concepto del reciclaje debe verse como un proceso de ``cuna a cuna'' en contraposición al pensamiento de ``la cuna a la tumba'' de hace unos años. En la práctica no hay una tumba para los materiales utilizados en las infraestructuras sostenibles actuales, tan sólo un nuevo inicio que permite la aplicación de tecnologías para lograr la meta de cero residuos en la rehabilitación y reconstrucción de pavimentos de adoquines.

Mientras que en el pasado los costes económicos eran el factor prioritario que empujaban al reciclaje, actualmente se tienen en cuenta los recursos limitados y la conciencia medioambiental en beneficio de la sostenibilidad en el proceso. Aunque el coste económico sigue siendo un factor importante, la concienciación política y social de la necesidad de ser sostenible ha aumentado considerablemente en los últimos años \cite{sustpave}. Los beneficios del reciclaje incluyen:

\begin{itemize}
  \item ahorros económicos.
  \item menor uso de materias primas no renovables.
  \item menor demanda de combustibles y emisiones asociadas del transporte de desechos a vertederos o de nuevos materiales a su destino.
  \item mejora del uso del suelo municipal, minimizando tanto la necesidad de vertederos como de nuevos espacios para la extracción de recursos.
\end{itemize}

\section{Introducción}
Una vez se han creado los modelos de fabricación, instalación y uso y mantenimiento, el \textbf{final de vida} es el último paso en el ciclo de vida. Para ello es necesario desarrollar un escenario de residuos –waste scenario—. Para la mayoría de los materiales utilizados SimaPro posee datos estándar, pero será de ayuda desarrollar un escenario propio simplificado para los residuos post-uso. Como se ha podido ver en la tabla \ref{desglosemateriasprimas} del capítulo \ref{cap:fabricacion}, el adoquín está compuesto de cuatro materiales:

\begin{enumerate}
  \item cemento.
  \item arena.
  \item árido.
  \item agua.
\end{enumerate}

Esto significa que el modelo de residuos deberá contener al menos los datos de fin de vida de estos cuatro elementos. Las características del escenario según los dispuesto en \cite{euroadoquin} son las siguientes:
\begin{itemize}
  \item el 95\% de los adoquines vuelven a ser reciclados en nuevos pavimentos,
  \item el 5\% son desechados en vertederos, asumiendo que estos vertederos modernos poseen un sistema de recolección apropiado.
\end{itemize}

% Tutorial Wood pdf

\section{Reciclaje}
\section{Procesos}


% 4.6.1 Structural inorganic building materials
% Inert materials like concrete and bricks etc. lead to no emissions in inert material landfills. After sorting in a building waste sorting plant, inert materials like cement from concrete can also be transferred to sanitary landfills as part of a landfilled fine fraction26. For inert material to sanitary landfill a composition from hydrated cement is adopted (cf. chapter 4.1.6 'Solidifying cement to residual material landfill' on page 22).

% 4.18.3 Wastewater from concrete production
% In the production of concrete, during the mixing of cement, water and gravel, a wastewater is produced that must be treated. Wastewater compositions are given in (Kellenberger et al. 2007). The chromium content is CrVI. In the wastewater treatment plant CrVI is partially converted to CrIII and transferred to sludge (50 w%). The remainder is emitted in the effluent as (soluble) CrVI. The wastewater is assumed to be disposed in a class 3 wastewater treatment plant.

% Busquedas:

% \cite{sustpave}

% https://www.google.es/search?client=safari&rls=en&q=reports+environmental+grimes&ie=UTF-8&oe=UTF-8&gws_rd=cr&ei=4nZiUqGGF8aVswbwroHoBw#q=lci+data+concrete+products&rls=en&spell=1
% https://www.google.es/search?client=safari&rls=en&q=reports+environmental+grimes&ie=UTF-8&oe=UTF-8&gws_rd=cr&ei=4nZiUqGGF8aVswbwroHoBw#q=reports+environmental+benefit+recycling+grimes&rls=en
