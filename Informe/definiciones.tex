%!TEX root = informe.tex
\chapter{Definiciones y abreviaturas}
\begin{itemize}
  \item AENOR (Asociación Española de Normalización y Certificación): entidad de certificación de sistemas de gestión, productos y servicios, y responsable del desarrollo y difusión de las normas UNE.
  \item Análisis del Ciclo de Vida (ACV): recopilación y evaluación de las entradas, las salidas y los impactos ambientales potenciales de un sistema del producto a través de su ciclo de vida.
  \item Análisis del Inventario del Ciclo de Vida (ICV): fase del análisis del ciclo de vida que implica la recopilación y la cuantificación de entradas y salidas para un sistema del producto a través de su ciclo de vida.
  \item Aspecto ambiental: elemento de las actividades, productos o servicios de una organización que puede interactuar con el medio ambiente.
  \item BUWAL: Bundesamt für Unwelt, Wald und Landshaft. Oficina Federal de Medio Ambiente, Bosque y Campo (Suiza).
  \item Categoría de impacto: clase que representa asuntos ambientales de interés a la cual se pueden asignar los resultados del análisis del inventario del ciclo de vida.
  \item CEN: Comité Europeo de Normalización.
  \item Ciclo de vida: etapas consecutivas e interrelacionadas de un sistema del producto, desde la adquisición de materia prima o de su generación a partir de recursos naturales hasta la disposición final.
  \item De la cuna a la tumba: expresión que referencia al ciclo de vida de un producto desde la extracción de las materias primas hasta la disposición.
  \item Eco-Indicador: indicador ambiental, desarrollado por PRé Consultants para el gobierno de Holanda.
  \item Ecodiseño: diseño que considera acciones orientadas a la mejora ambiental del producto o servicio en todas las etapas de su ciclo de vida, desde su creación en la etapa conceptual, hasta su tratamiento como residuo.
  \item Emisiones atmosféricas: introducción en la atmósfera por el hombre, de forma directa o indirecta, de sustancias o energía que tengan una acción perjudicial para la salud humana o el medio ambiente.
  \item Evaluación del Impacto del Ciclo de Vida (EICV): fase del análisis del ciclo de vida en la que los hallazgos del análisis del inventario o de la evaluación del impacto, o de ambos, se evalúan en relación con el objetivo.
  \item Impacto ambiental: alteración apreciable sobre la salud y bienestar de cualquier ser vivo o sobre el medio ambiente. En relación al ACV, se trata de la anticipación razonable a un efecto.
  \item ISO (International Standard Organitation). Organización Internacional de Estándares.
  \item Life Cycle Assessment (LCA): acrónimo en inglés de Análisis de Ciclo de Vida.
  \item Límite del sistema: conjunto de criterios que especifican cuales de los procesos unitarios son parte de un sistema del producto.
  \item Medio ambiente: conjunto de factores físico-químicos (agua, aire, clima, etc.), biológicos (fauna, flora y suelo) y socioculturales (asentamiento y actividad humana, uso y disfrute del territorio, formas de vida, etc.) que integran el entorno en que se desarrolla la vida del hombre y la sociedad (RD 4/1986 de 23 de enero 1986).
  \item Proceso unitario: elemento más pequeño considerado en el análisis del inventario del ciclo de vida para el cual se cuantifican datos de entrada y salida.
  \item Producto evitado: aquel material o producto que no es necesario generar debido a que ya se ha obtenido durante los procesos de otro producto.
  \item UNE: Una Norma Española.
  \item UNE-EN: Una Norma Española que además es Norma Europea (European Norm) a través del CEN.
  \item UNE-EN-ISO: Adaptación de normativa ISO a ámbito europeo por el CEN y de ahí al ámbito español por AENOR.
  \item Unidad funcional: aquella prestación o función que realiza un producto que permite su comparación con otros.
  \item Vida útil: duración estimada que un objeto puede tener cumpliendo correctamente con la función para la cual ha sido creado.
\end{itemize}
