%!TEX root = informe.tex
\chapter{Objeto}
El objetivo principal de este proyecto es el Análisis de Ciclo de Vida del adoquín común prefabricado del cemento utilizado en obras civiles y urbanismo. Se desea analizar el ciclo de vida del adoquín desde la obtención de la materia prima hasta su fin de vida.

Con este Análisis de Ciclo de Vida se pretende evaluar el comportamiento ambiental de las distintas etapas de su ciclo de vida, las cargas ambientales asociadas a estas etapas e identificar las posibles mejoras.

Para poder conseguir estos objetivos, se pueden establecer los siguientes objetivos básicos:
\begin{itemize}
\item Estudio y análisis de los procesos productivos, de instalación, mantenimiento y reciclado de los adoquines.
\item Análisis de las diferentes metodologías de Análisis de Ciclo de Vida.
\item Elección de una unidad de referencia del producto a partir de la cual se puedan normalizar los datos de entrada y salida y que permita su comparación con otros productos o con etapas de su ciclo de vida del mismo.
\item Desarrollo de un inventario de materiales y procesos de cada etapa del ciclo de vida.
\item Conocer y evaluar los impactos ambientales asociados a cada una de las etapas del producto y a su ciclo completo.
\end{itemize}

A su vez, la redacción del presente proyecto bajo la dirección del Departamento de Expresión Gráfica, Diseño y Proyectos de la Universidad de Málaga tiene como finalidad última la obtención del título de Ingeniero Industrial.
