%!TEX root = informe.tex
\chapter{Conclusiones}\label{cap:conclusiones}
\section{Conclusiones generales}
En este proyecto se ha realizado un Análisis de Ciclo de Vida de un adoquín prefabricado de cemento de dimensiones 200\times 100\times 60 \si{mm}, tomando como unidad funcional un metro cuadrado del mismo.

El estudio de ACV supone una potente herramienta para detectar los procesos con mayor carga ambiental y los impactos ocasionados al medioambiente de forma normalizada y objetiva. Una vez se localizan estos puntos calientes pueden comunicarse a la empresa para realizar mejoras en el proceso repercutiendo bien energéticamente, económicamente o como publicidad de tipo ambiental. Se puede recurrir al etiquetado ambiental \cite{iso14020} para estimular la demanda de productos y servicios con menores cargas ambientales indicando información relevante sobre su ciclo de vida —tal como la energía consumida o la cantidad de \ce{CO2} generado— y así satisfacer una demanda creciente de productos con información ambiental.

La carga ambiental mayor se produce en la fase del ciclo de vida de extracción de materias primas, fabricación e instalación, siendo la fase de uso y mantenimiento muy inferior aunque no despreciable y la de fin de vida tomada en consideración por los beneficios ambientales que aporta.

Los procesos unitarios donde se producen las mayores cargas ambientales son la extracción de materias primas para la capa bituminosa y el árido grueso de la capa base de la instalación y el proceso de fabricación del cemento Portland.

La fase del Análisis de Ciclo de Vida más compleja es la de Inventario de Ciclo de Vida, debido a que el inventario que ofrece la base de datos \textit{ecoinvent} no disponía de la amplia mayoria de procesos que intervienen en este proyecto y la información disponible era muy reducida. También ha aportado complejidad a la realización de este proyecto la baja disponibilidad de otros proyectos que sirvieran de base que utilizaran el ACV como método de estudio en este sector.

El software de análisis SimaPro ha simplificado en gran medida la fase de Evaluación del Inventario de Ciclo de Vida y la de Interpretación, no sólo por la incorporación del método ReCiPe, sino por su adaptación del flujo de trabajo a la normativa UNE-EN-ISO 14040:2006 y 14044:2006.


\section{Futuros trabajos y ampliaciones}

Este método de análisis desarrollado durante este proyecto es aplicable no sólo a adoquines sino a cualquier producto o servicio, por lo que se puede usar como base para otros productos prefabricados del cemento u otros tipos de pavimentos como el de asfalto o cerámico para poder hacer comparaciones entre ellos.

También se podría ampliar el estudio a otros países —donde cambiaría el mix eléctrico— u otros procesos alternativos de fabricación distintos al de este estudio.

Por último, un Análisis del Coste de Ciclo de Vida (Life Cycle Costing) del producto permitiría analizar todos los costes (directos e indirectos, variables y fijos) asignables al producto a lo largo de su ciclo de vida por o para cualquier agente asociado a las fases de la vida del producto (proveedor, productor, consumidor, etc.) y así poder decidir si las posibles mejoras ambiental obtenidas a partir del Análisis de Ciclo de Vida son también económicamente eficientes.
