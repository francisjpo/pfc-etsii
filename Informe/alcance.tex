%!TEX root = informe.tex
\chapter{Alcance}

En este proyecto se analiza la posibilidad de aprovechar la radiación solar, para la producción de energía eléctrica, en ubicaciones situadas en España y Rumanía, mediante centrales fotovoltaicas conectadas con la red eléctrica.

En el capítulo tres se presenta el consumo de energía eléctrica en el mundo. Para Es- paña se presenta también la evolución de la producción de energía eléctrica en función de su origen y se hace una previsión de la misma para el año 2008. Se hace una presentación de las energías renovables y de su nivel de implantación a nivel mundial, remarcando la energía solar fotovoltaica en España y Rumanía. A continuación, se presentan los principa- les componentes de una central fotovoltaica.

El capítulo cuatro muestra las fuentes de información utilizadas para la realización del proyecto.

En requisitos de diseño (capítulo 6) se presentan las dos ubicaciones inicialmente elegidas para el análisis: Tarragona y Bucarest.

El capitulo siete, análisis de soluciones, presenta los datos de radiación solar y tem- peratura para las dos ubicaciones elegidas inicialmente y también se buscan las ubicaciones con mayor radiación solar en España y Rumanía, presentando también en este caso la ra- diación solar y temperatura. Por último, se hace un análisis económico de las centrales fo- tovoltaicas emplazadas en cada ubicación para determinar el grado de rentabilidad de la inversión en cada caso. Finalmente, el capítulo ocho presenta las mejores soluciones, tanto técnicas como económicas.
