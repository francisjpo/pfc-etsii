%!TEX root = informe.tex
\chapter{Alcance}\label{cap:alcance}

El presente proyecto seguirá el siguiente proceso de estudio:

\begin{itemize}
  \item En primer lugar se establecen las características del producto a estudio, historia de su desarrollo y materias primas que lo componen.
  \item Se expondrá la relevancia del producto con el impacto medioambiental y se explicará la metodología de Análisis de Ciclo de Vida, su importancia y las herramientas con las que se trabajará.
  \item A continuación se realizará un Análisis de Ciclo de Vida del producto desde ``la cuna hasta la tumba'', es decir: extracción de materias primas, fabricación e instalación, uso y mantenimiento, y fin de vida.
  \item Una vez analizadas las distintas fases, se hará una comparación entre ellas.
  \item Por último, se desarrollarán las conclusiones obtenidas del estudio y posibles mejoras o futuras líneas de estudio.
\end{itemize}

El estudio se distribuirá de la siguiente manera:

\begin{itemize}
  \item En el capítulo \ref{cap:antecedentes} se presentará el escenario en el que aparecen los adoquines, impactos potenciales, materias primas y los procesos de instalación, mantenimiento y final de vida.
  \item En el capítulo \ref{cap:normas} se indicarán las normas de referencia para la redacción de este proyecto.
  \item En el capítulo \ref{cap:definiciones} se listarán los términos más importantes y las abreviaturas de uso en este proyecto.
  \item En el capítulo \ref{cap:metodologia_acv} se analizará la metodología del Análisis de Ciclo de Vida y sus fases, métodos de análisis, métodos para la evaluación de inventario y categorías de impacto, donde se establecerá la unidad funcional, los límites del sistema y las categorías de impacto a tener en cuenta.
  \item En el capítulo \ref{cap:acv_definicion} se establecerán los objetivos y el alcance del Análisis de Ciclo de Vida.
  \item En el capítulo \ref{cap:acv_inventario} se explicarán y modelarán los procesos que forman el inventario de todas las fases del ciclo de vida.
  \item En el capítulo \ref{cap:acv_evaluacion} se desarrollará la fase de evaluación del impacto del análisis de todas las fases y en conjunto.
  \item En el capítulo \ref{cap:acv_interpretacion} se verificarán los análisis de integridad, sensibilidad y coherencia, se extraerán conclusiones de la evaluación y se presentarán posibles soluciones o mejoras.
  \item En el capítulo \ref{cap:conclusiones} se presentarán conclusiones generales del análisis completo y del proceso de elaboración de este proyecto, así como futuras líneas de estudio.
\end{itemize}
