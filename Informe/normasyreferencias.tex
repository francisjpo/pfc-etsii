%!TEX root = informe.tex
\chapter{Normas y referencias}\label{cap:normas}
\section{Disposiciones legales y normas aplicadas}

Para la realización de este proyecto se han tenido en cuenta la siguiente normativa:

\begin{itemize}
  \item UNE-EN-ISO 14040:2006, Gestión Ambiental. Análisis de ciclo de vida. Principios y marco de referencia.
  \item UNE-EN-ISO 14440:2006, Gestion Ambiental. Análisis de ciclo de vida. Requisitos y directrices.
  \item UNE-EN-ISO 150041EX:1998, Análisis de ciclo de vida simplificado.
  \item UNE-EN-ISO 14006:2011, Sistemas de gestión ambiental. Directrices para la incorporación del ecodiseño.
  \item UNE-EN 1338:2004/AC:2006 Adoquines de hormigón. Especificaciones y métodos de ensayo.
  \item UNE-EN 197-1:2011 La norma europea de especificaciones de cementos comunes.
  \item UNE 80301:1996 Cementos. Cementos comunes. Composicion, especificaciones y criterios de conformidad.
  \item UNE 127338:2007 Propiedades y condiciones de suministro y recepción de los adoquines de hormigón. Complemento nacional a la Norma UNE EN 1338.
  \item UNE-CEN/TR 15941:2011 IN Sostenibilidad en la construcción. Declaraciones ambientales de producto. Metodología para la selección y uso de datos genéricos.
  \item UNE-EN15804:2012 Sostenibilidad en la construcción.Declaracionesambientales de producto.
  \item UNE-EN 15978:2012 Sostenibilidad en la construcción. Evaluación del comportamiento ambiental de los edificios. Métodos de cálculo.
  \item ISO. UNE-ISO 21930 Sostenibilidad en la construcción de edificios. Declaración ambiental de productos de construcción.
\end{itemize}

\bibliographystyle{plain}
\bibliography{informe}

\section{Programas de cálculo}

El software de Análisis de Ciclo de Vida elegido es SimaPro v7.3.3 de PRé Consultants. La base de datos de inventario es \textit{ecoinvent} v2.2.

Los cálculos se han realizado mediante la aplicación de hoja de cálculo Numbers.app de Apple.

\section{Plan de gestión de la calidad aplicado durante la redacción del Proyecto}

Para la realización de este proyecto se ha aplicado la siguiente normativa:
\begin{itemize}
  \item UNE 157001:2002 Criterios generales para la elaboración de proyectos.
  \item UNE 50132:1994 Documentación. Numeración de las divisiones y subdivisiones en los documentos escritos.
  \item UNE 1027:1995 Dibujos técnicos. Plegado de planos.
  \item UNE 1032:1982 Dibujos técnicos. Principios generales de representación.
  \item UNE 1035:1995 Dibujos técnicos. Cuadro de rotulación.
  \item UNE 1039:1994 Dibujos técnicos. Acotación. Principios generales, definiciones, métodos de ejecución e indicaciones especiales.
\end{itemize}

Durante la redacción de este proyecto se han corroborado los datos aportados por el fabricante. Se han revisado errores de transcripción de datos, fallos en los cálculos, así como errores gramaticales y ortográficos. Además, se ha comprobado la consistencia de los conceptos de estudio con la metodología empleada. Por último, se ha utilizado un sistema de copia de seguridad basada en control de versiones, por la cual los datos pueden ser recuperados o consultados en cualquier momento.
