%!TEX root = informe.tex
\chapter{Proceso de fabricación, almacenaje, suministro y recepción}
Manual Euroadoquín

Cómo se construyen los bloques de cemento

Las materias primas —cemento, arena, áridos y agua— se transportan hasta la planta de fabricación mediante camiones o ferrocarril. Actualmente los áridos y la arena ya no son apilados a bajo unos techados en las explanadas adyacentes a las plantas, sino que el propio transporte rellena las tolvas de forma automática.

Los áridos utilizados para producir adoquines puede incluir arena, gravilla y piedra de machaqueo si se pretende obtener un producto de peso normal. Si se desea que el adoquín sea más ligero —entre un 20 y un 45 \%— sin mermar sus propiedades estructurales se utilizan materiales como pizarra, arcilla, escoria de altos hornos y cenizas de carbón según su disponibilidad y coste.

Las tolvas tienen dosificadores que descargan la cantidad programada de materia prima sobre dos cintas transportadoras —una para áridos y arena, otra para cemento— con básculas de pesaje incorporadas que se comunican con el sistema de control y cortan el flujo de descarga.

La cinta de áridos descarga sobre un skip que eleva los materiales hasta una mezcladora. La cinta de cemento descarga directamente sobre la mezcladora.Previamente al añadido del agua se produce un ciclo de mezclado en seco. Para asegurar la consistencia del lote el agua es generalmente añadido mediante un sistema electrónico de control que dosifica el caudal. En el caso de que haya otros aditivos, tales como acelerantes o colorantes, es en este momento cuando se incorporan a la mezcla. Cuando se termina de añadir el agua se produce el mezclado creando hormigón fresco. El hormigón sale de la mezcladora mediante una cinta transportadora que contiene otra báscula de pesaje y se dirige hacia la tolva de hormigón que se encuentra en lo alto de la prensa, donde es dosificado en moldes.



\section{Dosificación}
\section{Amasado}
\section{Vibrocompresión}
\section{Curado}
\section{Embalaje y almacenamiento}
\section{Suministro}
\section{Recepción}
