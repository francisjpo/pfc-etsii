%!TEX root = informe.tex
\chapter{Análisis de Ciclo de Vida de un adoquín: Inventario del Ciclo de Vida}\label{cap:acv_inventario}

\section{Introducción}\label{sec:intro_icv}
Como se ha mencionado en la sección \ref{sec:etapaslca}, el Inventario del Ciclo de Vida (ICV) implica la recopilación de los datos y los procedimientos de cálculo para cuantificar las entradas y salidas pertinentes del sistema del producto. Es un proceso iterativo, ya que a medida que se recopilan datos, se aprende más sobre el sistema y se pueden identificar nuevos requisitos o limitaciones que requieran cambios en los procedimientos de recopilación de datos \cite{iso14040}. Los datos pueden clasificarse como:

\begin{itemize}
  \item las entradas de energía, materias primas, entradas auxiliares y otras entradas físicas;
  \item los productos, coproductos y residuos;
  \item las emisiones al aire, los vertidos al agua y suelo;
  \item otros aspectos ambientales.
\end{itemize}

Los datos para cada unidad de proceso del sistema del producto se obtienen bien de la industria o bien de bases de datos. En este caso se ha utilizado la base de datos \textit{ecoinvent} v2.2 (explicada en la sección \ref{sec:ecoinvent}).

Existen cinco etapas principales, de las cuales las tres primeras se agrupan en la etapa de fabricación:
\begin{itemize}
  \item Adquisición de materias primas: actividades necesarias para la extracción de las materias primas y aportaciones de energía del medio ambiente, incluyendo el transporte previo a la producción.
  \item Procesado de materias primas: actividades necesarias para convertir las materias primas en una forma que pueda ser utilizada para fabricar un producto.
  \item Fabricación y montaje: actividades necesarias para convertir los materiales en el producto deseado listado para ser transportado y distribuido.
  \item Uso y mantenimiento: utilización del producto acabado a lo largo de su vida en servicio, incluyendo materias primas y energía.
  \item Fin de vida: una vez que el producto ha servido para su función inicial y consecuentemente se recicla a través del mismo sistema de producto (ciclo cerrado de reciclaje) o entra en un nuevo sistema de producto (ciclo de reciclaje abierto), incluyendo energía y desechos.
\end{itemize}

\section{Fase de extracción de materias primas, fabricación e instalación}

\subsection{Bases del modelado para la extracción de materias primas y fabricación}\label{sec:basesfabricacion}
Cada adoquín mide 200x100x60 \si{mm} y tiene una masa de 3 \si{kg}. Cada bandeja está formada por 25 adoquines, por lo que son necesarias 2 bandejas para tener la \textit{Unidad funcional} de 1 \si{m^2}, con un total de 50 adoquines/\si{m^2} (ecuación \ref{eq:masa}):

\begin{gather}
200 mm \times 100 mm \times 25\ ud/bandeja \times 2\ bandejas = 1 m^2\\
3 kg/ud \times 50 ud/m^2 = 150 kg/m^2 \text{ de masa para adoquín}\label{eq:masa}
\end{gather}

Para disponer de 150 \si{kg/m^2} de masa para adoquín se aplican los porcentajes de materias primas sobre la fórmula base para adoquín ``Holanda 6'' proporcionada por el fabricante para obtener las masas de cada materia prima reflejadas en la tabla \ref{desglosemateriasprimas}.

\begin{table}[!htb]
\centering
\begin{tabular}{lcccc}
\toprule
\multicolumn{5}{c}{Consumo de materias primas por \si{m^2} de adoquín fabricado}\\
\midrule
Materia prima & \% Fórmula & Masa (\si{kg}) & Proced. & Dist. (\si{km})\\
\midrule
Árido tipo 5/7 & 37.75 & 56.63 & Alh. Torre & 8\\
Arena tipo 0/5 & 47.16 & 70.74 & Alh. Torre & 8\\
Cemento Portland 52.5N & 10.06 & 15.09 & Málaga-El Palo & 30\\
Agua & 5.03 & 7.54 & Red & -\\
\bottomrule
\end{tabular}
\caption{Desglose de materias primas por \si{m^2} de adoquín fabricado.}
\label{desglosemateriasprimas}
\end{table}

En el Anexo \ref{apend:catalogo} también se especifica un diagrama de Gantt de los procesos para una simulación realizada para fabricar 1 \si{m^2} de adoquín, además de los consumos energéticos desglosados.

\subsection{Cemento}
El cemento se transporta a granel en camiones con tanques a presión hasta la fábrica. Allí se almacena en silos provistos de compresores que descargan el material desde el tanque hasta su interior. El compresor es alimentado por electricidad mediante una toma de corriente conectada a la red eléctrica. La descarga del silo es únicamente por gravedad con válvulas dosificadoras de control de caudal (ver tabla \ref{modeladodelcemento}).

Las unidades para el modelado del camión (lorry) vienen expresadas en \si{kg\times km}, mientras que el uso del silo se proporciona en \si{m^3}, dada una densidad media del cemento Portland de 1250 \si{kg/m3} \cite{website:cement}.

\begin{gather}
15.09 kg \times 30 km = 453 kg\times km\\
15,09 kg / 1250 kg/m^3 = 0.0121 m^3
\end{gather}

El mix eléctrico se obtiene del consumo del compresor del silo proporcional a una cantidad de 15.09 \si{kg}, si la potencia del compresor son 30kW, velocidad de carga del silo es 35 \si{\tonne/h} para un tiempo de llenado de 35 minutos.

\begin{gather}
30 kW \times 1 h \times \frac{35 min}{60 min} = 17.5 kWh = 63 MJ \text{ para 20 toneladas}\\
35 t/h \times \frac{35 min}{60 min} = 20 t \text{ de cemento con el silo cargado}\\
15.09 kg \times \frac{63 MJ}{20 t} = 4.79 kJ
\end{gather}

\begin{table}[!htb]
\centering
\begin{tabular}{p{8cm}rc}
\toprule
\multicolumn{3}{c}{Cemento Portland CEM I 52.5Z gris}\\
\midrule
Materiales/ensamblajes & Cantidad & Unidad\\
\midrule
Portland cement, strength class Z 52.5, at plant/CH U & 15.09 & \si{kg}\\
\midrule
Procesos & Cantidad & Unidad\\
\midrule
Transport, lorry 16-32t, EURO4/RER U & 453 & \si{kg*km}\\
Tower silo, plastic/CH/I U & 0.0121 & \si{m^3}\\
Electricity mix 2013/ES U & 4.79 & \si{kJ}\\
\bottomrule
\end{tabular}
\caption{Modelado del cemento.}
\label{modeladodelcemento}
\end{table}

\subsection{Arena y áridos}
Las arenas y áridos se transportan hasta la planta de fabricación mediante camiones. Actualmente los áridos y la arena ya no se apilan a bajo techados en las explanadas adyacentes a las plantas, sino que el propio transporte rellena las tolvas de forma automática.

El tipo de arena que se utiliza en la planta es 0/5 —granulometría en milímetros de las partículas que forman la arena— no está directamente disponible en SimaPro. En su lugar, se ha optado por tomar el material ``Sand 0/2'' (Arena tipo 0/2), que además de pertenecer a la clasificación general de arena —de 0 a 5 \si{mm}—, la descripción de SimaPro indica que puede utilizarse como árido natural estándar en la industria de la construcción (ver tabla \ref{modeladodelaarena}).

\begin{quote}
Technical purpose of product or process: Standard mineral product used as natural aggregates in the construction industry according to the applied technology.
\end{quote}

Las unidades para el modelado del camión (lorry) vienen expresadas en \si{kg\times km}.

\begin{equation}
70.74 kg \times 8 km = 566 kg\times km
\end{equation}

\begin{table}[!htb]
\centering
\begin{tabular}{p{8cm}rc}
\toprule
\multicolumn{3}{c}{Arena tipo 0/5}\\
\midrule
Materiales/ensamblajes & Cantidad & Unidad\\
\midrule
Sand 0/2, wet and dry quarry, production mix, at plant, undried/RER S & 70.74 & \si{kg}\\
\midrule
Procesos & Cantidad & Unidad\\
\midrule
Transport, lorry 16-32t, EURO4/RER U & 566 & \si{kg*km}\\
\bottomrule
\end{tabular}
\caption{Modelado de la arena.}
\label{modeladodelaarena}
\end{table}

Los áridos utilizados para producir adoquines puede incluir arena, gravilla y piedra de machaqueo si se pretende obtener un producto de peso normal. Si se desea que el adoquín sea más ligero —entre un 20 y un 45 \%— sin mermar sus propiedades estructurales se utilizan materiales como pizarra, arcilla, escoria de altos hornos y cenizas de carbón según su disponibilidad y coste.

El tipo de árido utilizado en planta es de granulometría 5/7 —en milímetros—, catalogado como gravilla. No está directamente disponible en SimaPro, por lo que en su lugar, se ha optado por tomar el material ``Gravel, crushed'' (gravilla de machaqueo) (tabla \ref{modeladodearido}).

Las unidades para el modelado del camión (lorry) vienen expresadas en \si{kg\times km}.

\begin{equation}
56.63 kg \times 8 km = 566 kg\times km
\end{equation}

\begin{table}[!htb]
\centering
\begin{tabular}{p{8cm}rc}
\toprule
\multicolumn{3}{c}{Árido tipo 5/7}\\
\midrule
Materiales/ensamblajes & Cantidad & Unidad\\
\midrule
Gravel, crushed, at mine/CH U & 56.63 & \si{kg}\\
\midrule
Procesos & Cantidad & Unidad\\
\midrule
Transport, lorry 16-32t, EURO4/RER U & 453 & \si{kg*km}\\
\bottomrule
\end{tabular}
\caption{Modelado del árido.}
\label{modeladodearido}
\end{table}

\subsection{Agua}
La mayoría de las plantas tienen una fuente de agua municipal (\textit{tap water}) que proporciona potable perfectamente válida para el uso en la fabricación de hormigón (ver tabla \ref{modeladodelagua}).

\begin{table}[!htb]
\centering
\begin{tabular}{p{8cm}rc}
\toprule
\multicolumn{3}{c}{Agua}\\
\midrule
Materiales/ensamblajes & Cantidad & Unidad\\
\midrule
Tap water, at user/RER U & 7.54 & \si{kg}\\
\bottomrule
\end{tabular}
\caption{Modelado del agua.}
\label{modeladodelagua}
\end{table}

\subsection{Mix eléctrico}

El mix eléctrico es la cesta energética de un país, es decir, la combinación de las diferentes fuentes de energía que cubren su suministro eléctrico. Es un indicador de las fuentes energéticas que se emplean para producir electricidad. Cuanto más bajo es el mix, mayor es la contribución de fuentes energéticas bajas en carbono.

Disponer de un mix eléctrico actualizado ayuda a obtener unos resultados más precisos en el Análisis de Ciclo de Vida. La base de datos de \textit{ecoinvent} proporciona un modelo no actualizado para España, sobretodo teniendo en cuenta los últimos avances en adopción de energías renovables en el sector energético español. De esta forma, se ha añadido un nuevo modelo (tabla \ref{modeladomixelectrico}) con las cifras actualizadas al presente año para 1 \si{kWh} \cite{mlgceballos}.

\begin{table}[!htb]
\centering
\begin{tabular}{p{8cm}rc}
\toprule
\multicolumn{3}{c}{Electricity mix 2013/ES U}\\
\midrule
Materiales/combustibles & Cantidad & Unidad\\
\midrule
Electricity, hard coal, at power plant/ES U & 0.0596 & \si{kWh}\\
Electricity, lignite, at power plant/ES U & 0.0268 & \si{kWh}\\
Electricity, oil, at power plant/ES U & 0.0527 & \si{kWh}\\
Electricity, natural gas, at power plant/ES U & 0.3025 & \si{kWh}\\
Electricity, industrial gas, at power plant/ES U & 0.0147 & \si{kWh}\\
Electricity, hydropower, at power plant/ES U & 0.131 & \si{kWh}\\
Electricity, hydropower, at pumped storage power plant/ES U & 0.020 & \si{kWh}\\
Electricity, nuclear, at power plant/UCTE U & 0.205 & \si{kWh}\\
Electricity, production mix photovoltaic, at plant/ES U & 0.024 & \si{kWh}\\
Electricity, at wind power plant/RER U & 0.1454 & \si{kWh}\\
Electricity, at cogen with biogas engine, allocation exergy/CH U & 0.0146 & \si{kWh}\\
Electricity, production mix FR/FR U & 0.0056 & \si{kWh}\\
\bottomrule
\end{tabular}
\caption[Modelado del mix eléctrico español en 2013.]{Modelado del mix eléctrico español en 2013. Fuente: \cite{mlgceballos}.}
\label{modeladomixelectrico}
\end{table}

\subsection{Dosificador de arena y áridos}

El sistema de control central manda una señal a los dosificadores para que viertan la cantidad ordenada por el programa principal de fabricación. Dichos dosificadores consisten en una especie de tolva con forma de embudo con un cierre controlado por el sistema.

No existe un modelo de referencia en SimaPro para este tipo de dosificadores —\textit{feed hopper}, en inglés—, por lo que se ha simplificado un modelo válido \cite{woodpellet}. Se le ha añadido la parte de mix eléctrico de los datos del fabricante (Apéndice \ref{apend:datos}). El dosificador del cemento y del agua están incluidos en su propio modelo (silo y abastecimiento de la red respectivamente).

\begin{table}[!htb]
\centering
\begin{tabular}{p{8cm}rc}
\toprule
\multicolumn{3}{c}{Dosificadores para arena y áridos}\\
\midrule
Materiales/combustibles & Cantidad & Unidad\\
\midrule
Steel, low-alloyed, at plant/RER U & 47 & \si{kg}\\
\midrule
Electricidad/calor & Cantidad & Unidad\\
\midrule
Electricity mix 2013/ES U & 0.0021 & \si{MJ}\\
\bottomrule
\end{tabular}
\caption{Modelado de los dosificadores para arena y áridos.}
\label{modeladodedosificadores}
\end{table}

\subsection{Cintas transportadoras de áridos y cemento}

Las tolvas descargan la cantidad programada de materia prima sobre dos cintas transportadoras —una para áridos y arena, otra para cemento— con básculas de pesaje incorporadas que se comunican con el sistema de control y cortan el flujo de descarga.

La cinta de áridos descarga sobre un skip que eleva los materiales hasta una mezcladora. La cinta de cemento descarga directamente sobre la mezcladora.

Las distancias están medidas sobre planos (Apéndice \ref{planos}), y los consumos se han obtenido de los ensayos en fábrica a partir de la potencia de la cinta transportadora y el tiempo de funcionamiento (Potencia=Energía/Tiempo).

\begin{table}[!htb]
\centering
\begin{tabular}{p{8cm}rc}
\toprule
\multicolumn{3}{c}{Cinta transportadora para arena y áridos}\\
\midrule
Materiales/combustibles & Cantidad & Unidad\\
\midrule
Conveyor belt, at plant/RER/I U & 14.6 & \si{m}\\
\midrule
Electricidad/calor & Cantidad & Unidad\\
\midrule
Electricity mix 2013/ES U & 0.1827 & \si{MJ}\\
\bottomrule
\end{tabular}
\caption{Modelado de la cinta transportadora para arena y áridos.}
\label{modeladodecintaarena}
\end{table}

\begin{table}[!htb]
\centering
\begin{tabular}{p{8cm}rc}
\toprule
\multicolumn{3}{c}{Cinta transportadora para cemento}\\
\midrule
Materiales/combustibles & Cantidad & Unidad\\
\midrule
Conveyor belt, at plant/RER/I U & 7.3 & \si{m}\\
\midrule
Electricidad/calor & Cantidad & Unidad\\
\midrule
Electricity mix 2013/ES U & 0.0975 & \si{MJ}\\
\bottomrule
\end{tabular}
\caption{Modelado de la cinta transportadora para cemento.}
\label{modeladodecintacemento}
\end{table}

\subsection{Skip y mezcladora}

Para asegurar la consistencia del lote el agua se añade mediante un sistema electrónico de control que dosifica el caudal. En el caso de que haya otros aditivos, tales como acelerantes o colorantes, es en este momento cuando se incorporan a la mezcla. Cuando se termina de añadir el agua se produce el mezclado creando hormigón fresco.

La base de datos \textit{ecoinvent} proporciona un modelo para el mezclado del hormigón, ``Paster mixing'' en el que se introduce la masa de la mezcla, 150 \si{kg} y al que se le añade la parte de mix eléctrico de los datos del fabricante (Apéndice \ref{apend:datos}).

\begin{table}[!htb]
\centering
\begin{tabular}{p{8cm}rc}
\toprule
\multicolumn{3}{c}{Skip y mezcladora}\\
\midrule
Materiales/combustibles & Cantidad & Unidad\\
\midrule
Plaster mixing/CH U & 150 & \si{kg}\\
\midrule
Electricidad/calor & Cantidad & Unidad\\
\midrule
Electricity mix 2013/ES U & 1.71 & \si{MJ}\\
\bottomrule
\end{tabular}
\caption{Modelado del skip y la mezcladora.}
\label{modeladoskip}
\end{table}

\subsection{Cinta transportadora para hormigón}

El hormigón sale de la mezcladora mediante una cinta transportadora que contiene otra báscula de pesaje y se dirige hacia la tolva de hormigón que se encuentra en lo alto de la prensa.

La base de datos \textit{ecoinvent} proporciona un modelo para la cinta transportadora, ``Conveyor belt'' en el que se introduce la distancia de recorrido, 12.1 \si{m}, y al que se le añade la parte de mix eléctrico de los datos del fabricante (Apéndice \ref{apend:datos}).

\begin{table}[!htb]
\centering
\begin{tabular}{p{8cm}rc}
\toprule
\multicolumn{3}{c}{Cinta transportadora para hormigón}\\
\midrule
Materiales/combustibles & Cantidad & Unidad\\
\midrule
Conveyor belt, at plant/RER/I U & 12.1 & \si{m}\\
\midrule
Electricidad/calor & Cantidad & Unidad\\
\midrule
Electricity mix 2013/ES U & 0.088 & \si{MJ}\\
\bottomrule
\end{tabular}
\caption{Modelado de la cinta transportadora para hormigón.}
\label{modeladodecintahormigon}
\end{table}

\subsection{Tolva para hormigón}

La tolva de hormigón se encarga de dosificar el hormigón en el molde de la prensa. No existe un modelo de referencia en SimaPro para tolvas —\textit{hopper}, en inglés—, por lo que se ha simplificado un modelo válido \cite{foodnottrash}. Se le ha añadido la parte de mix eléctrico de los datos del fabricante (Apéndice \ref{apend:datos}).

\begin{table}[!htb]
\centering
\begin{tabular}{p{8cm}rc}
\toprule
\multicolumn{3}{c}{Tolva para hormigón}\\
\midrule
Materiales/combustibles & Cantidad & Unidad\\
\midrule
Steel, low-alloyed, at plant/RER U & 470 & \si{kg}\\
\midrule
Electricidad/calor & Cantidad & Unidad\\
\midrule
Electricity mix 2013/ES U & 0.0161 & \si{MJ}\\
\bottomrule
\end{tabular}
\caption{Modelado de la tolva para hormigón.}
\label{modeladotolvahormigon}
\end{table}

\subsection{Vibrocompresión}

La tolva dosifica el hormigón fresco, que cae en los moldes para adoquines. Los moldes tienen una longevidad muy alta —aproximadamente un millón de ciclos de prensado— y su durabilidad depende de las propiedades abrasivas de los áridos utilizados.

El molde se compone de dos partes: la parte donde se inyecta el hormigón (hembra) y la parte que se coloca encima para dar forma (macho). La prensa tiene incorporado un carro alimentador encargado de proporcionar la parte hembra. Se inyecta el hormigón en el molde hembra, el molde macho baja con la prensa y el hormigón es compactado y cimentado usando un sistema combinado de presión y vibración. Cada molde puede producir 25 adoquines de 200x100x60\si{\milli\meter}, lo que proporciona a una superficie adoquinada de 0.5\si{\square\meter}. Los adoquines son moldeados de una sola pieza y extraidos del molde inmediatamente después de la vibro-compresión sobre una bandeja de madera.

SimaPro no proporciona un modelo para prensas de cemento. Estudiando las similitudes entre una planta de hormigón —\textit{concrete plant}— y la planta objeto de este proyecto, se puede aproximar un modelo de la prensa basado en la maquinaria del primero \cite{buildingproducts}. La aproximación ``Industrial machine, heavy, unspecified, at plant'' pide la masa de la máquina industrial no específica que realiza el proceso. De acuerdo al fabricante, ese dato es de 1380 \si{kg}, al que se le ha añadido la parte de mix eléctrico también proporcionado por el fabricante (Apéndice \ref{apend:datos}).

\begin{table}[!htb]
\centering
\begin{tabular}{p{8cm}rc}
\toprule
\multicolumn{3}{c}{Prensado}\\
\midrule
Materiales/combustibles & Cantidad & Unidad\\
\midrule
Industrial machine, heavy, unspecified, at plant/RER/I U & 9000 & \si{kg}\\
\midrule
Electricidad/calor & Cantidad & Unidad\\
\midrule
Electricity mix 2013/ES U & 0.437 & \si{MJ}\\
\bottomrule
\end{tabular}
\caption{Modelado del prensado.}
\label{modeladoprensado}
\end{table}

\subsection{Cinta transportadora para piezas frescas}

La bandeja con las piezas frescas es trasladada sobre un transportador de rodillos hasta un ascensor.

SimaPro no proporciona un modelo para este tipo de transportador. Estudiando las similitudes entre un transportador de rodillos y una cinta transportadora se puede aproximar un modelo propio basado en que la principal diferencia es la falta de una banda de rodadura (tabla \ref{modeladotransportadorrodillos}).

\begin{table}[!htb]
\centering
\begin{tabular}{p{8cm}rc}
\toprule
\multicolumn{3}{c}{Transportadora de rodillos para piezas frescas}\\
\midrule
Materiales/combustibles & Cantidad & Unidad\\
\midrule
Concrete, sole plate and foundation, at plant/CH U & 0.01 & \si{m^3}\\
Section bar rolling, steel/RER U & 500 & \si{kg}\\
Steel, low-alloyed, at plant/RER U & 530 & \si{kg}\\
Transport, lorry >16t, fleet average/RER U & 55.5 & \si{\tonne\times km}\\
Wire drawing, steel/RER U & 29.6 & \si{kg}\\
\midrule
Residuos y emisiones para tratamiento & Cantidad & Unidad\\
\midrule
Disposal, building, reinforced concrete, to final disposal/CH U & 23 & \si{kg}\\
Disposal, steel, 0\% water, to municipal incineration/CH U & 29.6 & \si{kg}\\
\bottomrule
\end{tabular}
\caption{Modelado de 1 metro de transportador de rodillos.}
\label{modeladotransportadorrodillos}
\end{table}

La aproximación ``Roller conveyor, at plant'' pide como parámetro la distancia de recorrido, 6.8 \si{m}, al que se le ha añadido la parte de mix eléctrico también proporcionado por el fabricante (Apéndice \ref{apend:datos}).

\begin{table}[!htb]
\centering
\begin{tabular}{p{8cm}rc}
\toprule
\multicolumn{3}{c}{Transportadora de rodillos para piezas frescas}\\
\midrule
Materiales/combustibles & Cantidad & Unidad\\
\midrule
Conveyor belt, at plant/RER/I U & 6.8 & \si{m}\\
\midrule
Electricidad/calor & Cantidad & Unidad\\
\midrule
Electricity mix 2013/ES U & 0.077 & \si{MJ}\\
\bottomrule
\end{tabular}
\caption{Modelado del transportador de rodillos para piezas frescas.}
\label{modeladotransportadorpiezas}
\end{table}

\subsection{Ascensor}

Este ascensor tiene diez alturas, de forma que cada vez que recibe una bandeja con adoquines frescos, la bandeja anterior sube una altura y monta la siguiente. El ascensor se encarga de alimentar un carro multiforca de diez alturas.

SimaPro no proporciona un modelo para un ascensor de estas características, por lo que se ha obtado por un elemento genérico que sí esté en la base de datos de \textit{ecoinvent} como ``Industrial machine, heavy, unspecified, at plant'' que pide la masa de la máquina industrial no específica que realiza el proceso. De acuerdo al fabricante, ese dato es de 320 \si{kg}, al que se le ha añadido la parte de mix eléctrico también proporcionado por el fabricante (Apéndice \ref{apend:datos}).

\begin{table}[!htb]
\centering
\begin{tabular}{p{8cm}rc}
\toprule
\multicolumn{3}{c}{Ascensor}\\
\midrule
Materiales/combustibles & Cantidad & Unidad\\
\midrule
Industrial machine, heavy, unspecified, at plant/RER/I U & 320 & \si{kg}\\
\midrule
Electricidad/calor & Cantidad & Unidad\\
\midrule
Electricity mix 2013/ES U & 0.126 & \si{MJ}\\
\bottomrule
\end{tabular}
\caption{Modelado del ascensor.}
\label{modeladodelascensor}
\end{table}

\subsection{Multiforca}
Cuando las diez alturas está ocupadas se cargan en un carro multiforca —\textit{rack transporter}, en inglés— automatizado que transporta las piezas hasta un secadero.

Las piezas permanecen en el secadero curándose a temperatura ambiente entre 24 y 48 horas.

Una vez transcurrido el tiempo de curado, los adoquines están secos y listos para ser recogidos por otro carro multiforca automatizado que recoge las bandejas y las lleva a un descensor.

SimaPro tampoco proporciona un modelo para un carro multiforca, por lo que también se ha obtado por un elemento genérico que sí esté en la base de datos de \textit{ecoinvent} como ``Industrial machine, heavy, unspecified, at plant'' que pide la masa de la máquina industrial no específica que realiza el proceso. De acuerdo al fabricante, un carro multiforca se compone de un tres partes: transportador de forca (rackveyor), rodadura (crawler) y vehículo (transfer car), sumando en total 7500 \si{kg}, a lo que se le ha añadido la parte de mix eléctrico también proporcionado por el fabricante (Apéndice \ref{apend:datos}).

\begin{table}[!htb]
\centering
\begin{tabular}{p{8cm}rc}
\toprule
\multicolumn{3}{c}{Multiforca}\\
\midrule
Materiales/combustibles & Cantidad & Unidad\\
\midrule
Industrial machine, heavy, unspecified, at plant/RER/I U & 7500 & \si{kg}\\
\midrule
Electricidad/calor & Cantidad & Unidad\\
\midrule
Electricity mix 2013/ES U & 0.516 & \si{MJ}\\
\bottomrule
\end{tabular}
\caption{Modelado de la multiforca.}
\label{modeladomultiforca}
\end{table}

\subsection{Descensor}
El descensor coloca las bandejas con los adoquines secos en un transportador de rodillos. Su modelado es el mismo que el del ascensor.

\begin{table}[!htb]
\centering
\begin{tabular}{p{8cm}rc}
\toprule
\multicolumn{3}{c}{Descensor}\\
\midrule
Materiales/combustibles & Cantidad & Unidad\\
\midrule
Industrial machine, heavy, unspecified, at plant/RER/I U & 320 & \si{kg}\\
\midrule
Electricidad/calor & Cantidad & Unidad\\
\midrule
Electricity mix 2013/ES U & 0.126 & \si{MJ}\\
\bottomrule
\end{tabular}
\caption{Modelado del descensor.}
\label{modeladodeldescensor}
\end{table}

\subsection{Transporte de bandejas hasta paletizadora}
El transportador de rodillos lleva las bandejas hasta una paletizadora para hacer bloques de hasta cinco alturas. Se ha vuelto a utilizar el modelado de la tabla \ref{modeladotransportadorrodillos}, introduciendo los 6.88 \si{m} de recorrido entre el origen y el destino, a lo que se le ha añadido la parte de mix eléctrico también proporcionado por el fabricante (Apéndice \ref{apend:datos}).

\begin{table}[!htb]
\centering
\begin{tabular}{p{8cm}rc}
\toprule
\multicolumn{3}{c}{Transporte de bandejas hasta paletizadora}\\
\midrule
\multicolumn{2}{c}{Materiales/combustibles}\\
\cmidrule(r){1-2}
Descripción & Cantidad & Unidad\\
\midrule
Roller conveyor, at plant/RER/I U & 6.88 & \si{m}\\
\midrule
\multicolumn{2}{c}{Electricidad/calor}\\
\cmidrule(r){1-2}
Descripción & Cantidad & Unidad\\
\midrule
Electricity mix 2013/ES U & 0.021 & \si{MJ}\\
\bottomrule
\end{tabular}
\caption{Modelado del transporte de bandejas hasta paletizadora.}
\label{modeladobandejaspalet}
\end{table}

\subsection{Paletizado y flejado}

La paletizadora dispone los bloques de adoquín sobre un pallet para su posterior almacenaje y transporte. De esta forma se consigue una mayor uniformidad y facilidad de manipulación de la carga, ahorrando espacio y rentabilizando los tiempos de carga—descarga y manipulación.

La paletizadora impulsa el pallet hasta la flejadora que aplica varias lazadas de flejes para evitar que los adoquines se desprendan del conjunto.

El proceso de SimaPro ``Packing, clay products'' abarca ambos procesos en un único modelo, en el que se introduce la masa de la carga, 150 \si{km}, a lo que se le ha añadido la parte de mix eléctrico proporcionado por el fabricante (Apéndice \ref{apend:datos}).

\begin{table}[!htb]
\centering
\begin{tabular}{p{8cm}rc}
\toprule
\multicolumn{3}{c}{Flejado y paletizado}\\
\midrule
Materiales/combustibles & Cantidad & Unidad\\
\midrule
Packing, clay products/CH U & 150 & \si{kg}\\
\midrule
Electricidad/calor & Cantidad & Unidad\\
\midrule
Electricity mix 2013/ES U & 0.065 & \si{MJ}\\
\bottomrule
\end{tabular}
\caption{Modelado del flejado y paletizado.}
\label{modeladodelflejadoypaletizado}
\end{table}

\begin{table}[!htb]
\centering
\begin{tabular}{p{8cm}rc}
\toprule
\multicolumn{3}{c}{Transporte de pallets hasta flejadora}\\
\midrule
Materiales/combustibles & Cantidad & Unidad\\
\midrule
Roller conveyor, at plant/RER/I U & 3.1 & \si{m}\\
\midrule
Electricidad/calor & Cantidad & Unidad\\
\midrule
Electricity mix 2013/ES U & 0.021 & \si{MJ}\\
\bottomrule
\end{tabular}
\caption{Modelado del transporte de pallets hasta flejadora.}
\label{modeladopalletsflejadora}
\end{table}

\subsection{Transporte de pallets flejados hasta zona de recogida}

La flejadora descansa los conjuntos paletizados sobre un un transportador de rodillos para ser posteriormente llevados a almacén.

\begin{table}[!htb]
\centering
\begin{tabular}{p{8cm}rc}
\toprule
\multicolumn{3}{c}{Transporte de pallets flejados hasta zona de recogida}\\
\midrule
Materiales/combustibles & Cantidad & Unidad\\
\midrule
Roller conveyor, at plant/RER/I U & 39.1 & \si{m}\\
\midrule
Electricidad/calor & Cantidad & Unidad\\
\midrule
Electricity mix 2013/ES U & 0.0315 & \si{MJ}\\
\bottomrule
\end{tabular}
\caption{Modelado del transporte de pallets flejados hasta zona de recogida.}
\label{modeladopalletsrecogida}
\end{table}

\subsection{Transporte de pallets hasta almacén}
Finalmente, un toro de almacén (forklift truck) transporta cada pallet de adoquines a la zona de almacenaje, a la espera de que los pedidos salgan de almacén.

SimaPro no incorpora en ninguna de sus bases de datos un modelo aproximado de un toro, generando el modelo de la tabla \ref{modeladoforklift}. El modelo comprende el habitáculo, horquilla, motor, batería, neumáticos y su parte proporcional de trabajo de mecanizado y ensamblado \cite{ecocosts}.

\begin{table}[!htb]
\centering
\begin{tabular}{p{8cm}rc}
\toprule
\multicolumn{3}{c}{Forklift truck}\\
\midrule
Materiales/combustibles & Cantidad & Unidad\\
\midrule
Steel, low-alloyed, at plant/RER U & 2250 & \si{kg}\\
Lead, primary, at plant/GLO U & 1200 & \si{kg}\\
Sulfuric acid, at plant/kg/RNA & 2800 & \si{kg}\\
Acrylonitrile-butadiene-styrene copolymer, ABS, at plant/RER U & 90 & \si{kg}\\
Copper, at regional storage/RER U & 50 & \si{kg}\\
Turning, steel, conventional, primarily roughing/RER S & 1000 & \si{kg}\\
Drilling, CNC, steel/RER U & 1250 & \si{kg}\\
Copper wire, technology mix, consumption mix, at plant, cross section 1 mm² EU-15 S & 50 & \si{kg}\\
\bottomrule
\end{tabular}
\caption{Modelado de un toro de almacén (forklift truck).}
\label{modeladoforklift}
\end{table}

Al igual que no incluye un modelo de toro, tampoco incluye el transporte mediante un toro de almacén, por lo que se ha establecido un modelado basado en el de un furgón de carga de menos de 3.5 \si{\tonne} para añadir las partes proporcionales de uso del toro, mantenimiento, asfalto y costes de operación (tabla \ref{modeladotransporteforklift}).

\begin{table}[!htb]
\centering
\begin{tabular}{p{8cm}rc}
\toprule
\multicolumn{3}{c}{Transport, forklift truck}\\
\midrule
Materiales/combustibles & Cantidad & Unidad\\
\midrule
Operation, van < 3,5t/RER U & 5.3015 & \si{km}\\
Forklift truck & 0.000024098 & p\\
Maintenance, van < 3.5t/RER/I U & 0.000024098 & p\\
Road/CH/I U & 0.0067419 & \si{my}\\
Operation, maintenance, road/CH/I U & 0.0062138 & \si{my}\\
\midrule
Residuos y emisiones & Cantidad & Unidad\\
Disposal, van < 3.5t/CH/I U & 0.000024098 & p\\
Disposal, road/RER/I U & 0.0067419 & \si{my}\\
\midrule
\bottomrule
\end{tabular}
\caption{Modelado del transporte con toro de almacén.}
\label{modeladotransporteforklift}
\end{table}

De esta forma, el modelado del transporte con el toro hasta el almacén se genera introduciendo las toneladas por kilómetro que se transportan (tabla \ref{modeladotransportetorito}).

\begin{table}[!htb]
\centering
\begin{tabular}{p{8cm}rc}
\toprule
\multicolumn{3}{c}{Transporte de pallets con torito hasta almacén}\\
\midrule
Materiales/combustibles & Cantidad & Unidad\\
\midrule
Transport, forklift truck/RER U & 0.0132 & \si{\tonne\times km}\\
\bottomrule
\end{tabular}
\caption{Modelado del transporte de pallets con torito hasta almacén.}
\label{modeladotransportetorito}
\end{table}

\subsection{Control informatizado}

Todo el sistema está centralizado en dos ordenadores que cargan los programas de funcionamiento situados en una cabina supervisada por un operario.

\begin{table}[!htb]
\centering
\begin{tabular}{p{8cm}rc}
\toprule
\multicolumn{3}{c}{Control informatizado}\\
\midrule
\multicolumn{2}{c}{Materiales/combustibles}\\
\cmidrule(r){1-2}
Descripción & Cantidad & Unidad\\
\midrule
Desktop computer, without screen, at plant/GLO U & 2 & p\\
Keyboard, standard version, at plant/GLO U & 2 & p\\
LCD flat screen, 17 inches, at plant/GLO U & 2 & p\\
Mouse device, optical, with cable, at plant/GLO U & 2 & p\\
Network access devices, internet, at user/CH/I U & 2 & p\\
Router, IP network, at server/CH/I U & 1 & p\\
Power supply unit, at plant/CN U & 2 & p\\
\midrule
\multicolumn{2}{c}{Electricidad/calor}\\
\cmidrule(r){1-2}
Descripción & Cantidad & Unidad\\
\midrule
Electricity mix 2013/ES U & 0.486 & \si{MJ}\\
\bottomrule
\end{tabular}
\caption{Modelado del control informatizado.}
\label{modeladodecontrol}
\end{table}

\subsection{Iluminación}

La iluminación del recinto se compone principalmente de 60 tubos fluorescentes de 40 \si{W}. Debido a que SimaPro no tiene modelados tubos CFL, se ha añadido un modelo a la base de datos \cite{cflbulb}.

\begin{table}[!htb]
\centering
\begin{tabular}{p{8cm}rc}
\toprule
\multicolumn{3}{c}{Iluminación}\\
\midrule
Materiales/combustibles & Cantidad & Unidad\\
\midrule
CFL Bulb 40W & 60 & p\\
\multicolumn{2}{c}{Desglose para 1 p. CFL Bulb 40W}\\
\cmidrule(r){1-2}
Aluminium alloy, AlMg3, at plant/RER U & 7.09 & \si{g}\\
Oriented polypropylene film E & 4.25 & \si{g}\\
Iron-nickel-chromium alloy, at plant/RER U & 6.27 & \si{g}\\
Copper wire, technology mix, consumption mix, at plant, cross section 1 \si{mm^2} EU-15 S & 4.25 & \si{g}\\
41 Plastics basic, virgin, EU27 & 1.42 & \si{g}\\
Integrated circuit, IC, logic type, at plant/GLO U & 1.42 & \si{g}\\
\midrule
Electricidad/calor & Cantidad & Unidad\\
\midrule
Electricity mix 2013/ES U & 0.972 & \si{MJ}\\
\bottomrule
\end{tabular}
\caption{Modelado de la iluminación.}
\label{modeladodeiluminacion}
\end{table}

\subsection{Limpieza de la mezcladora y molde}

Una vez finalizada la producción se procede a la limpieza de la mezcladora y el molde con agua y vaciando el contenido (tabla \ref{modeladolimpiezamezcladora}).

\begin{table}[!htb]
\centering
\begin{tabular}{p{8cm}rc}
\toprule
\multicolumn{3}{c}{Limpieza de la mezcladora y molde}\\
\midrule
Materiales/combustibles & Cantidad & Unidad\\
\midrule
Tap water, at user/RER/I U & 100 & \si{kg}\\
\bottomrule
\end{tabular}
\caption{Modelado la limpieza de la mezcladora y molde.}
\label{modeladolimpiezamezcladora}
\end{table}

\subsection{Bases para el modelado de la instalación}

Debido a que hay múltiples tipos de vía y uso destinado, se ha optado para el presente proyecto modelar la instalación más común, \textbf{arterias principales}, que pertenece a la \textbf{categoría de tráfico C1} y una \textbf{calidad de explanada E2} con una base granular. Con esta clasificación, siguiendo las recomendaciones de \cite{euroadoquinc} el corte del terreno 1 \si{m^2} de superficie de terreno, que es la Unidad Funcional, será el reflejado en la tabla \ref{cortedelterreno}.

\begin{table}[!htb]
\centering
\begin{tabular}{lrrr}
\toprule
\multicolumn{4}{c}{Capas componentes para arterías principales C1-E2 con base granular}\\
\midrule
Capa componente & Grosor (\si{cm}) & Densidad (\si{kg/m^3}) & Volumen (\si{m^3})\\
\midrule
Adoquín \& Sellado & 10 & 2650 & 0.1\\
Lecho de árido & 4 & 1650 & 0.04\\
Base granular & 20 & 2560 & 0.2\\
Subbase & — & — & —\\
Explanada & \multicolumn{3}{c}{Aplanar y compactar}\\
\midrule
Total & 34 & — & 0.34\\
\bottomrule
\end{tabular}
\caption{Capas componentes para arterías principales C1-E2 con base granular.}
\label{cortedelterreno}
\end{table}

\subsection{Árido grueso para base granular (zahorra)}

Para la base granular es recomendable utilizar áridos calizos, y evitar en cualquier caso el uso de áridos con contenido en arcilla —arena de miga, arcillas refractarias—.

El acabado de la base debe ser similar al exigido para una superficie destinada a carreteras, usando una imprimación bituminosa. Tras compactar la base es recomendable hacer un sellado por medio de betún de curado rápido o emulsiones bituminosas para poder evitar filtraciones de agua a través de las juntas y que éstas dañen la base durante los primeros meses de la instalación.

La arena caliza —\textit{Limestone}, en inglés— aparece en la base de datos de SimaPro. Se pide como parámetro la masa de arena que se empleará para el rellenado. Si se tiene en cuenta que la base tiene una profundidad de 20 \si{cm} y un área de 1 \si{m^2}:

\begin{gather}
\text{Volumen de arena} = 0.2 \times 1 = 0.2 m^3\\
\rho_{arena}=2560 kg/m^3\\
\text{Masa de arena} = 0.2 \times 2560 = 512 kg
\end{gather}

Por otro lado, si se supone una distancia de entrega de 50 km hasta el destino de la instalación en un camión de transporte, se tiene:

\begin{equation}
512 kg \times 50 km = 25600 kg \times km
\end{equation}

\begin{table}[!htb]
\centering
\begin{tabular}{p{8cm}rc}
\toprule
\multicolumn{3}{c}{Arena para base granular}\\
\midrule
Materiales/ensamblajes & Cantidad & Unidad\\
\midrule
Limestone, milled, packed, at plant/CH U & 512 & \si{kg}\\
\midrule
Procesos & Cantidad & Unidad\\
\midrule
Transport, lorry 16-32t, EURO4/RER U & 25600 & \si{kg\times km}\\
\bottomrule
\end{tabular}
\caption{Modelado de la arena para base granular.}
\label{modeladoarenabase}
\end{table}

Igualmente, para la capa bituminosa de 5 cm que se aplica:

\begin{gather}
\text{Volumen de betún} = 0.05 \times 1 = 0.05 m^3\\
\rho_{betún}=1100 kg/m^3\\
\text{Masa de betún} = 0.05 \times 1100 = 55 kg
\end{gather}

Si se supone que el material proviene de una distancia de entrega de 50 km hasta el destino de la instalación en un camión de transporte, se tiene:

\begin{equation}
55 kg \times 50 km = 2750 kg \times km
\end{equation}

\begin{table}[!htb]
\centering
\begin{tabular}{p{8cm}rc}
\toprule
\multicolumn{3}{c}{Capa bituminosa para base granular}\\
\midrule
Materiales/ensamblajes & Cantidad & Unidad\\
\midrule
Bitumen sealing, at plant/RER U & 55 & \si{kg}\\
\midrule
Procesos & Cantidad & Unidad\\
\midrule
Transport, lorry 16-32t, EURO4/RER U & 2750 & \si{kg\times km}\\
\bottomrule
\end{tabular}
\caption{Modelado de la capa bituminosa para base granular.}
\label{modeladocapabituminosa}
\end{table}

\subsection{Árido semi-fino para lecho de arena}

La capa para el lecho de arena debe estar formada por áridos de resistencia geomecánica elevada, preferentemente de machaqueo ya que presentan mayores ángulos que mejoran la cohesión de la capa.

En general, los áridos deben ser poco finos, limpios y libres de elementos contaminantes.

La arena caliza —\textit{Limestone}, en inglés— aparece en la base de datos de SimaPro. Se pide como parámetro la masa de arena que se empleará para el rellenado. Si se tiene en cuenta que la base tiene una profundidad de 20 \si{cm} y un área de 1 \si{m^2}:

\begin{gather}
\text{Volumen de árido} = 0.04 \times 1 = 0.04 m^3\\
\rho_{arena}=1650 kg/m^3\\
\text{Masa de árido} = 0.04 \times 1650 = 66 kg
\end{gather}

Por otro lado, si se supone una distancia de entrega de 50 km hasta el destino de la instalación en un camión de transporte, se tiene:

\begin{equation}
66 kg \times 50 km = 3300 kg \times km
\end{equation}

\begin{table}[!htb]
\centering
\begin{tabular}{p{8cm}rc}
\toprule
\multicolumn{3}{c}{Árido semi-fino para lecho de arena}\\
\midrule
Materiales/ensamblajes & Cantidad & Unidad\\
\midrule
Gravel, crushed, at mine/CH U & 66 & \si{kg}\\
\midrule
Procesos & Cantidad & Unidad\\
\midrule
Transport, lorry 16-32t, EURO4/RER U & 3300 & \si{kg\times km}\\
\bottomrule
\end{tabular}
\caption{Modelado del árido semi-fino para lecho de arena.}
\label{modeladoaridosemifino}
\end{table}

\subsection{Arena para sellado}

La arena para sellado debe ser una arena sin exceso de finos, ya que si existen demasiados finos, se producirá un vaciado de las juntas con el uso y limpieza del pavimento o bien se filtrarás hacia el lecho.

Además debe ser libre de sales solubles y otros contaminantes, ya que pueden provocar la aparición de eflorescencias —igual que en el caso del lecho de árido—.

No se debe usar mortero para el sellado de las juntas, ya que no se podrán retirar para hacer tareas de mantenimiento —principal ventaja de los adoquines de hormigón—, además de perder flexibilidad del conjunto.

La arena de sílice —\textit{Silica sand}, en inglés— es un material adecuado para esta tarea \cite{website:ecoinvent}. SimaPro pide como parámetro la masa de arena que se empleará para el rellenado. Si se tiene en cuenta que el adoquín mide 10 \si{cm} de altura, 1 \si{cm} quedará enterrado en el lecho, y se añadirá 1 \si{cm} por encima para el vibrado; por otro lado, la junta medirá 5 \si{mm}, se puede obtener el volumen y a partir de él la masa:

\begin{gather}
1 m^2 = 50 \text{ adoquines} = 10 \times 5\\
Adoquin = 20 \times 10 \times 6 cm\\
\delta_{junta} = 0.5cm\\
Largo = 10 \times 10 + 9 \times 0.5 = 104.5 cm\\
Ancho = 5 \times 20 + 4 \times 0.5 = 102 cm\\
\text{Superficie de arena} = 104.5 \times 0.5 + 102 \times 0.5 = 103.25 cm^2\\
\text{Volumen de arena} = 103.25 \times 10 = 1032.5 cm^3\\
\rho_{arena}=2.6 g/cm^3\\
\text{Masa de arena} = 1032.5 \times 2.65 = 2.74 kg
\end{gather}

Por otro lado, si se supone una distancia de entrega de 50 km hasta el destino de la instalación en un camión de transporte, se tiene:

\begin{equation}
2.74 kg \times 50 km = 137 kg \times km
\end{equation}

\begin{table}[!htb]
\centering
\begin{tabular}{p{8cm}rc}
\toprule
\multicolumn{3}{c}{Arena para sellado}\\
\midrule
Materiales/ensamblajes & Cantidad & Unidad\\
\midrule
Silica sand, at plant/DE U & 2.74 & \si{kg}\\
\midrule
Procesos & Cantidad & Unidad\\
\midrule
Transport, lorry 16-32t, EURO4/RER U & 137 & \si{kg\times km}\\
\bottomrule
\end{tabular}
\caption{Modelado de la arena para sellado.}
\label{modeladoarenasellado}
\end{table}

\subsection{Excavación del terreno}

A la hora de realizar un pavimento de adoquines, se debe realizar en primer lugar la excavación del terreno. Dado que el grosor total de las capas componentes es de 34 \si{cm} y se dispone de un área de 1 \si{m^2}, el volumen a introducir para el modelo será 0.34 \si{m^3}, utilizando como entrada de SimaPro de excavación con herramienta hidráulica, \textit{Excavation, hydraulic digger}.

\begin{table}[!htb]
\centering
\begin{tabular}{p{8cm}rc}
\toprule
\multicolumn{3}{c}{Excavación del terreno}\\
\midrule
Materiales/combustibles & Cantidad & Unidad\\
\midrule
Excavation, hydraulic digger/RER U & 0.34 & \si{m^3}\\
\bottomrule
\end{tabular}
\caption{Modelado de la excavación del terreno.}
\label{modeladoexcavacion}
\end{table}

\subsection{Compactación de la explanada}

Una vez excavado el terreno es necesario compactar lo que será la explanada. La bibliografía consultada no ha ofrecido ninguna solución óptima a este tipo de proceso, por lo que se ha optado por asemejar el tipo de trabajo de un tractor usando un rodillo para cultivar la tierra —\textit{Tillage, rolling}, en inglés— con el rodillo utilizado por una apisonadora o compactadora —\textit{road roller}, en inglés—. En este caso, la compactación se realiza en unidades de área, por lo que se ha introducido 1 \si{m^2} de superficie.

\begin{table}[!htb]
\centering
\begin{tabular}{p{8cm}rc}
\toprule
\multicolumn{3}{c}{Compactación de la explanada}\\
\midrule
Materiales/combustibles & Cantidad & Unidad\\
\midrule
Tillage, rolling/CH U & 1 & \si{m^2}\\
\bottomrule
\end{tabular}
\caption{Modelado de la compactación de la explanada.}
\label{modeladoexplanada}
\end{table}

\subsection{Compactación de la capa base}

En el caso de arterias principales no existe una capa subbase, por lo que se procederá a la extensión y compactación de la capa base. Una correcta ejecución es fundamental ya que esta capa es el principal elemento portante de la estructura y se encarga de transmitir hacia la explanada las cargas verticales. El espesor de esta base debe ser uniforme.

Es muy importante que el plano de la capa base respete una pendiente mínima del 1\% para permitir un drenaje adecuado de las aguas superficiales sin que provoquen daños a las capas portantes, y así evitar daños en la superficie.

La bibliografía consultada no ha ofrecido ninguna solución óptima a este tipo de proceso, por lo que se ha optado por asemejar el tipo de trabajo de un tractor usando un rodillo para cultivar la tierra —\textit{Tillage, rolling}, en inglés— con el rodillo utilizado por una apisonadora o compactadora —\textit{road roller}, en inglés—. En este caso, la compactación se realiza en unidades de área, por lo que se ha introducido 1 \si{m^2} de superficie.

\begin{table}[!htb]
\centering
\begin{tabular}{p{8cm}rc}
\toprule
\multicolumn{3}{c}{Compactación de la capa base}\\
\midrule
Materiales/combustibles & Cantidad & Unidad\\
\midrule
Tillage, rolling/CH U & 1 & \si{m^2}\\
\bottomrule
\end{tabular}
\caption{Modelado de la compactación de la capa base.}
\label{modeladocapabase}
\end{table}

\subsection{Compactación del lecho de árido}

El lecho de árido es, junto con la calidad del adoquín, el elemento fundamental que determina el comportamiento y durabilidad del pavimento. El lecho se extiende directamente sobre la capa base.

Una de las funciones principales es la de absorber las pequeñas diferencias de espesor de los adoquines siguiendo las tolerancias de la normativa \cite{une1338}, de forma que, una vez se hace la compactación de los adoquines, formen un plano de rodadura uniforme que transmita las cargas del tráfico sin deteriorar las piezas.

Otra de las funciones del lecho de árido es la de actuar como elemento de relleno inferior de las juntas de los adoquines. Al ser compactados los adoquines, quedan incrustados en el lecho, y así se evita el contacto directo entre las caras laterales de las piezas.


Al igual que en las compactaciones anteriores, la bibliografía consultada no ha ofrecido ninguna solución óptima a este tipo de proceso, por lo que se ha optado por asemejar el tipo de trabajo de un tractor usando un rodillo para cultivar la tierra —\textit{Tillage, rolling}, en inglés— con el rodillo utilizado por una apisonadora o compactadora —\textit{road roller}, en inglés—. En este caso, la compactación se realiza en unidades de área, por lo que se ha introducido 1 \si{m^2} de superficie.

\begin{table}[!htb]
\centering
\begin{tabular}{p{8cm}rc}
\toprule
\multicolumn{3}{c}{Compactación del lecho de árido}\\
\midrule
Materiales/combustibles & Cantidad & Unidad\\
\midrule
Tillage, rolling/CH U & 1 & \si{m^2}\\
\bottomrule
\end{tabular}
\caption{Modelado de la compactación del lecho de árido.}
\label{modeladolecho}
\end{table}

\subsection{Sellado con arena y vibrado del pavimento}\label{sec:selladoinstalacion}

Una vez colocados y alineados los adoquines de forma que el lecho de árido también sirva de separador entre las juntas, se extiende sobre el pavimento una capa ligera de arena para completar el espacio.

La importancia de esta etapa yace en que un relleno completo de las juntas hace que tanto esta arena como el árido del leche sean los transmisores de los esfuerzos laterales entre los adoquines. Si el pavimento soporta tráfico sin haber sido bien sellado, se pueden producr daños importantes sobre el mismo.

El sellado consiste en extender arena fina y seca sobre el pavimento e introducirla entre las juntas con un barrido manual o mecánico, intentando que quede una ligera capa de excedente sobre toda la superficie.

A continuación se realiza un proceso de compactación sobre el pavimento para garantizar un relleno adecuado de las juntas. Esta compactación se puede realizar bien con placas vibrantes —\textit{vibratory plates} o \textit{plate compactor}, en inglés— o con rodillos mecánicos con vibración. La fuerza vibratoria y el peso de las herramientas deben ser proporcionales al tipo de pavimento que se está ejecutando.

Una vez se realiza la compactación, el pavimento puede ponerse en servicio inmediatamente.

El proceso de compactación no aparece reflejado en las librerías que proporciona SimaPro. De acuerdo a \cite{rieradevall}, las placas vibrantes para compactar pueden modelarse mediante su consumo de combustible. En concreto 2.43 \si{MJ} de gasoil para 1 ft$^2$. Por lo que, extrapolando para 1 \si{m^2}, se tendría:

\begin{equation}
2.43\frac{MJ}{{ft}^2} \times \frac{10.764{ft^2}}{1m^2}=26.16MJ/m^2
\end{equation}

\begin{table}[!htb]
\centering
\begin{tabular}{p{8cm}rc}
\toprule
\multicolumn{3}{c}{Vibrado del pavimento}\\
\midrule
Materiales/combustibles & Cantidad & Unidad\\
\midrule
Diesel, burned in building machine/GLO U & 26.16 & \si{MJ}\\
\bottomrule
\end{tabular}
\caption{Modelado del vibrado del pavimento.}
\label{modeladovibrado}
\end{table}

\subsection{Limpieza final}

Cuando se ha terminado el vibrado del pavimento y se ha observado que las juntas quedan completamente rellenas, se debe iniicar el proceso de limpieza de la superficie para eliminar la arena de sellado excedente.

Esta limpieza se realiza mediante un barrido manual, dejando una pequeña capa de arena sobre el pavimento para que el tráfico termine de colocar sobre las juntas de forma natural. Una vez finalizada la limpieza se puede abrir la vía al uso destinado.

Este proceso no se ha incluido en SimaPro ya que es un trabajo manual de un operario y no se consumen directamente materiales o combustibles.

\subsection{Modelado completo de la extracción de materias primas, fabricación e instalación}\label{sec:modeladofabricacion}

Los modelos de los procesos explicados anteriormente han sido introducidos en SimaPro para crear un modelo completo que represente la fase de extracción de materias primas, fabricación e instalación de un metro cuadrado de adoquín modelo Holanda 6.

Aunque la mayoría de los procesos sólo ocurren una única vez, hay varios procesos que necesitan funcionar dos veces, debido a que son dos bandejas las que hay que preparar para fabricar el metro cuadrado (ver sección \ref{sec:basesfabricacion}) o bien, como en el caso de la multiforca, porque hay una ida y una vuelta hacia la zona de curación. El listado completo de procesos se muestra en las tablas \ref{modeladocompletofabricacionmaterias} y \ref{modeladocompletofabricacionprocesos}.

\begin{table}[!htp]
\centering
\begin{tabular}{p{8cm}rc}
\toprule
\multicolumn{3}{c}{Extracción, fabricación e instalación de 1 \si{m^2} adoquín Holanda 6}\\
\midrule
Materiales/ensamblajes & Cantidad & Unidad\\
\midrule
Agua & 1 & p\\
Arena tipo 0/5 & 1 & p\\
Árido tipo 5/7 & 1 & p\\
Cemento Portland CEM I 52.5Z gris & 1 & p\\
Árido grueso para base granular (zahorra) & 1 & p\\
Capa bituminosa para base granular (zahorra) & 1 & p\\
Árido semi-fino para lecho de arena & 1 & p\\
Arena para sellado de juntas & 1 & p\\
\bottomrule
\end{tabular}
\caption{Modelado de materias primas de la extracción, fabricación e instalación.}
\label{modeladocompletofabricacionmaterias}
\end{table}

\begin{table}[!htp]
\centering
\begin{tabular}{p{8cm}rc}
\toprule
\multicolumn{3}{c}{Extracción, fabricación e instalación de 1 \si{m^2} adoquín Holanda 6}\\
\midrule
Procesos & Cantidad & Unidad\\
\midrule
Dosificador de arena & 1 & p\\
Dosificador de áridos & 1 & p\\
Cinta transp. para arena y áridos & 1 & p\\
Cinta transp. para cemento & 1 & p\\
Skip+mezcladora & 1 & p\\
Cinta transp. para hormigón & 1 & p\\
Tolva para hormigón & 2 & p\\
Prensado & 2 & p\\
Transportador de rodillos para piezas frescas & 2 & p\\
Ascensor & 1 & p\\
Multiforca & 2 & p\\
Descensor & 1 & p\\
Transporte de bandejas hasta paletiz. & 1 & p\\
Paletizado+flejado & 1 & p\\
Transporte de pallets hasta flejadora & 1 & p\\
Transporte de pallets flejados hasta zona de recogida & 1 & p\\
Transporte de pallets con torito hasta almacén & 1 & p\\
Control informatizado & 1 & p\\
Iluminación & 1 & p\\
Excavación del terreno para arteria principal C1-E2 & 1 & p\\
Compactación de la explanada & 1 & p\\
Compactación de la capa base & 1 & p\\
Compactación del lecho de arena & 1 & p\\
Sellado+vibrado del pavimento & 1 & p\\
\bottomrule
\end{tabular}
\caption{Modelado de los procesos de la extracción, fabricación e instalación.}
\label{modeladocompletofabricacionprocesos}
\end{table}

\section{Fase de uso y mantenimiento}\label{sec:faseusoymantenimiento}

\subsection{Bases para el modelado del uso y mantenimiento}
El pavimento de adoquines puede presentar tres situaciones durante su vida útil en las que intervengan recursos adicionales para su uso y mantenimiento:
\begin{itemize}
\item limpieza.
\item sellado de juntas.
\item apertura del pavimento por rotura o infraestructura urbana.
\end{itemize}

En los dos primeros casos se puede establecer un modelo debido a que existe un patrón periódico sobre el que diseñarlo. Sin embargo, el tercero no es posible establecer un modelo sencillo que prevea ese comportamiento, por lo que se procederá a idealizar el uso y mantenimiento del mismo sin roturas.

\subsection{Limpieza mensual}
Si se tiene en cuenta lo establecido en la sección \ref{sec:consideracionesuso}, se riega el pavimento una vez al mes en el que se consumen 5 litros de agua por cada metro cuadrado de adoquín instalado. Para una vida útil de 30 años, serán 359 limpiezas mensuales.

\begin{table}[!htb]
\centering
\begin{tabular}{p{8cm}rc}
\toprule
\multicolumn{3}{c}{Limpieza mensual}\\
\midrule
Materiales/combustibles & Cantidad & Unidad\\
\midrule
Tap water, at user/RER/I U & 5 & \si{kg}\\
\bottomrule
\end{tabular}
\caption{Modelado de la limpieza de mensual del pavimento.}
\label{modeladoagualimpiezamensual}
\end{table}

\subsection{Sellado de juntas}

Si se supone una labor de mantenimiento periódica cada 5 años de sellado de juntas, y la vida útil se considera de 30 años (sección \ref{sec:consideracionesuso}), se realizarán 5 sellados de juntas en el transcurso de su vida, ya que la sexta vez sería para renovar el pavimento con nuevos adoquines.

Para el cálculo se puede utilizar lo expuesto en la sección \ref{sec:selladoinstalacion}, utilizando la misma arena —arena silícea— y procedimiento de instalación —vibrado del pavimento. En este caso, se supondrá que las pérdidas de arena no van a ser del 100\% sino del 50\% de la masa inicial de arena de sellado instalada, transportada en camión una distancia de 50 \si{km}.

\begin{table}[!htb]
\centering
\begin{tabular}{p{8cm}rc}
\toprule
\multicolumn{3}{c}{Arena para sellado para mantenimiento}\\
\midrule
Materiales/ensamblajes & Cantidad & Unidad\\
\midrule
Silica sand, at plant/DE U & 1.37 & \si{kg}\\
\midrule
Procesos & Cantidad & Unidad\\
\midrule
Transport, lorry 16-32t, EURO4/RER U & 68.5 & \si{kg\times km}\\
\bottomrule
\end{tabular}
\caption{Modelado de la arena para sellado para mantenimiento.}
\label{modeladoarenaselladomantenimiento}
\end{table}

\begin{table}[!htb]
\centering
\begin{tabular}{p{8cm}rc}
\toprule
\multicolumn{3}{c}{Vibrado del pavimento para uso y mantenimiento}\\
\midrule
Materiales/combustibles & Cantidad & Unidad\\
\midrule
Diesel, burned in building machine/GLO U & 26.16 & \si{MJ}\\
\bottomrule
\end{tabular}
\caption{Modelado del vibrado del pavimento para uso y mantenimiento.}
\label{modeladovibradouso}
\end{table}

\subsection{Modelado completo del uso y mantenimiento}

Los modelos de los procesos explicados en la sección \ref{sec:faseusoymantenimiento} han sido introducidos en SimaPro para crear un modelo completo que represente el uso y mantenimiento de un metro cuadrado de adoquín modelo Holanda 6. El listado completo de procesos se muestra en la tabla \ref{modeladocompletousoymantenimiento}.

\begin{table}[!htb]
\centering
\begin{tabular}{p{8cm}rc}
\toprule
\multicolumn{3}{c}{Uso y mantenimiento de 1 \si{m^2} adoquín Holanda 6}\\
\midrule
Materiales/ensamblajes & Cantidad & Unidad\\
\midrule
Arena para sellado para mantenimiento & 1 & p\\
Agua para limpieza mensual & 359 & p\\
\midrule
Procesos & Cantidad & Unidad\\
\midrule
Sellado+vibrado del pavimento & 5 & p\\
\bottomrule
\end{tabular}
\caption{Modelado completo del uso y mantenimiento.}
\label{modeladocompletousoymantenimiento}
\end{table}

\section{Fase de fin de vida}\label{sec:fasefindevida}

\subsection{Bases para el modelado del fin de vida}
Una vez se han creado los modelos de fabricación, instalación y uso y mantenimiento, el \textbf{final de vida} es el último paso en el ciclo de vida. Para ello es necesario desarrollar un escenario de residuos –waste scenario—. SimaPro ofrece datos estándar para la mayoría de los materiales usados mediante sus bases de datos, pero será de ayuda desarrollar un escenario propio simplificado para los residuos post-uso.

Como se ha podido ver en la tabla \ref{desglosemateriasprimas} de la sección \ref{sec:basesfabricacion}, el adoquín está compuesto de cuatro materiales: cemento, arena, árido y agua. Esto significa que el modelo de residuos deberá contener al menos los datos de fin de vida de estos cuatro elementos. Las características del escenario según los dispuesto en \cite{euroadoquin} son las siguientes:
\begin{itemize}
  \item el 95\% de los adoquines son recuperados para su reciclaje;
  \item el 5\% son desechados en vertederos junto con otros residuos.
\end{itemize}

\subsection{Desensamblaje de los adoquines}

El pavimento de adoquines es retirado normalmente de forma manual por operarios especializados. Tras mucho tiempo sometidos al tráfico rodado, suelen estar estrechamente ligados unos con otros, por lo que se procede a abrir un área inicial para acceder al lecho de árido y a partir de ahí retirar los adoquines colindantes.

\begin{table}[!htb]
\centering
\begin{tabular}{p{8cm}rc}
\toprule
\multicolumn{3}{c}{Desensamblaje de los adoquines}\\
\midrule
Materiales/ensamblajes & Cantidad & Unidad\\
\midrule
T1 m2 adoquín Holanda 6 & 1 & p\\
\midrule
Procesos & Cantidad & Unidad\\
\midrule
Transport, lorry 16-32t, EURO4/RER U & 7130 & \si{kg\times km}\\
\bottomrule
\end{tabular}
\caption{Modelado del desensamblaje de adoquín.}
\label{modeladodesensamblaje}
\end{table}

\subsection{Escenarios de residuos}

Ese 95\% es —del total de 150 \si{kg} que forman el metro cuadrado— retirado de la zona donde fue instalado de forma manual —por lo que no se incorporará al inventario— y transportado hasta una planta de reciclaje de hormigón a una distancia supuesta de 50 \si{km} en un camión de 16—32 \si{\tonne} que cumple una normativa europea de emisiones EURO 4.

\begin{table}[!htb]
\centering
\begin{tabular}{p{8cm}rc}
\toprule
\multicolumn{3}{c}{Escenario de residuos para adoquín}\\
\midrule
Materiales/combustibles & Cantidad & Unidad\\
\midrule
Transport, lorry 16-32t, EURO4/RER U & 7500 & \si{kg\times km}\\
\midrule
Escenario de residuo/tratamiento & Porcentaje & Comentarios\\
\midrule
Reciclado de adoquín en áridos & 95\% & \\
Vertedero para adoquín & 5\% & \\
\bottomrule
\end{tabular}
\caption{Modelado del escenario de residuos para adoquín.}
\label{modeladoescenarioresiduos}
\end{table}

\subsection{Tratamiento de residuos}
\subsubsection{Reciclado de adoquín en áridos}
El proceso de reciclje consiste en la llegada en camión a la planta de reciclaje, donde es introducido en una tolva de recepción, donde se produce un machacado para convertirlo en árido (producto evitado) \cite{monografia,gerd}.

\begin{table}[!htb]
\centering
\begin{tabular}{p{8cm}p{2cm}c}
\toprule
\multicolumn{3}{c}{Reciclado de adoquín en áridos}\\
\midrule
\multicolumn{3}{c}{Materiales y/o tipos de residuo separados del flujo de residuos}\\
\midrule
Escenario de residuo/tratamiento & Material/tipo de residuo & Porcentaje\\
\midrule
Reciclaje de adoquín en áridos & Concrete block, at plant/DE U & 100\%\\
\midrule
\multicolumn{3}{c}{Materiales y/o tipos de residuo separados del flujo de residuos}\\
\midrule
Escenario de residuo/tratamiento & Porcentaje & \\
\midrule
Disposal, building, concrete, not reinforced, to final disposal/CH U & 100\% & \\
\bottomrule
\end{tabular}
\caption{Modelado del reciclado de adoquín en áridos.}
\label{modeladorecicladoenaridos}
\end{table}

\subsubsection{Vertedero para adoquín}
El 5\% residual de adoquines que aparece en mezclado con otros residuos urbanos en vertederos. Las causas de su aparición suelen ser excedentes abandonados de obras y pequeñas roturas de pavimentos recogidas por los servicios de limpieza.

\begin{table}[!htb]
\centering
\begin{tabular}{p{8cm}p{2cm}c}
\toprule
\multicolumn{3}{c}{Vertedero para adoquín}\\
\midrule
\multicolumn{3}{c}{Materiales y/o tipos de residuo separados del flujo de residuos}\\
\midrule
Escenario de residuo/tratamiento & Material/tipo de residuo & Porcentaje\\
\midrule
Vertedero de adoquines & Concrete block, at plant/DE U & 100\%\\
\midrule
\multicolumn{3}{c}{Materiales y/o tipos de residuo separados del flujo de residuos}\\
\midrule
Escenario de residuo/tratamiento & Porcentaje & \\
\midrule
Disposal, building, concrete, not reinforced, to final disposal/CH U & 100\% & \\
\bottomrule
\end{tabular}
\caption{Modelado del vertedero para adoquín.}
\label{modeladovertederoadoquin}
\end{table}

\subsection{Modelado completo del fin de vida}

Los modelos de los procesos explicados en la sección \ref{sec:fasefindevida} han sido introducidos en SimaPro para crear un modelo completo que represente el fin de vida de un metro cuadrado de adoquín modelo Holanda 6. El listado completo de procesos se muestra en la tabla \ref{modeladocompletofindevida}.

\begin{table}[!htb]
\centering
\begin{tabular}{p{8cm}rc}
\toprule
\multicolumn{3}{c}{Fin de vida de 1 \si{m^2} adoquín Holanda 6}\\
\midrule
Materiales/ensamblajes & Cantidad & Unidad\\
\midrule
1 m2 adoquín Holanda 6 & 1 & p\\
\midrule
Escenarios de residuos/disposición & & \\
\midrule
Escenarios de residuos para adoquín & & \\
\bottomrule
\end{tabular}
\caption{Modelado completo del fin de vida.}
\label{modeladocompletofindevida}
\end{table}
