\chapter{Prefabricados del cemento. Adoquines}
\section{Descripción general}
A lo largo de la historia de la humanidad se han ido utilizando diferentes tipos de adoquines para pavimentar los suelos urbanos \cite{eadmt04}. Los primeros adoquines eran de piedra, obtenidos a partir de los guijarros de río colocados sobre una capa de arena, usando una mezcla de cal y arena como sellante de juntas.

Debido al coste y el ruido del tráfico rodado, en la primera mitad del siglo XIX comenzaron a usarse los adoquines de madera, utilizando para el sellado residuos bituminosos. Debido a su reducida duración y a la posterior aparición de los neumáticos, los adoquines de madera son sustituidos por un modelo cerámico, con el que se usaba la misma arena tanto para la base como sellante.

Los adoquines de piedra seguían siendo más resistentes y además no eran tan deslizantes como los cerámicos, por lo que a finales del siglo XIX se comenzó la fabricación de los adoquines de hormigón. Estos proporcionaban una mayor uniformidad que los de piedra, eran muy resistentes y con un coste inferior. Alemania y Holanda fueron los primeros en incorporar este nuevo formato de adoquín a sus núcleos urbanos. Al principio se usaban modelos que imitaban a los de piedra tanto en forma como colocación, pero pronto se añadieron formas dentadas o curvas, permitiendo una mejor alineación con el trazado.

Finalmente, durante la década de los 70 se mejoraron sustancialmente los sistemas de fabricación, permitiendo una gran variedad de modelos de adoquines y un abaratamiento de los costes de fabricación e instalación.

\subsection{Ventajas del uso de adoquines}

En comparación con otros tipos de pavimentos tales como los asfálticos o los pavimentos contínuos hormigonados, los adoquines presentan las siguientes ventajas:

\begin{itemize}
\item Fabricación: no se utilizan derivados del petróleo, que suelen ser caros y contaminantes, además de requerir una mayor aportación de energía durante el proceso de fabricación. En contraposición, pueden utilizarse cementos y áridos locales, disminuyendo los costes de transporte.

El proceso de fabricación de los adoquines requiere una maquinaria específica debido a que son sometidos a presión y vibración para segurar una resistencia y durabilidad adecuadas. Esto implica un control sobre la fabricación, consistencia y fiabilidad del producto mayor que el resto de pavimentos.

\item Instalación: aunque los adoquines pueden colocarse de forma automatizada, están diseñados de base para ser colocados manualmente, permitiendo instalarse en zonas de difícil acceso, cargas elevadas (muelles de carga, aeropuertos, \ldots), resolver trazados complejos o pendientes pronunciadas. A diferencia de los pavimentos asfálticos, su ejecución no depende de la temperatura ambiente y pueden ser utilizados inmediatamente después de su finalización, lo que implica una reducción en los tiempos de ejecución de obra.

\item Comportamiento: los adoquines pueden ser diseñados para ser muy resistentes tanto a cargas verticales (distribuidas o puntuales) como a esfuerzos horizontales (aceleración-frenada, giros,\ldots). Además, soportan bien sin degradarse los vertidos de aceites y combustibles sobre el pavimento. Los niveles de ruido generados por el tráfico son similares o inferiores a otros pavimentos en ausencia de humedad y sensiblemente inferiores en condiciones de humedad, especialmente a bajas velocidades. La resistencia a deslizamiento es mayor al del resto de pavimentos.

\item Mantenimiento: la vida útil del adoquín viene determinada principalmente por el comportamiento de la base, subbase y explanada y no por el propio adoquín. La vida útil de cálculo suele ser a 30 años, aunque en condiciones normales puede superar los 50 años. De esta manera, al renovar el pavimento se pueden reutilizar entre un 90 y un 95\% de los adoquines originales \cite{eadmt04}. El adoquín es la mejor opción en zonas donde aún no se han implantado todos los servicios de públicos debido a que pueden ser levantados fácilmente para llevar tareas de instalación o reparación en el subsuelo. La conservación de los adoquines se limita al relleno de juntas erosionadas con arena de sellado cada cierto tiempo y a la reposición de adoquines fracturados.

\item Costes: aunque inicialmente el precio del metro cuadrado instalado es algo superior a otros pavimentos, a largo plazo es mucho más barato debido al menor mantenimiento y la reutilización de piezas. Los pavimentos asfálticos y hormigonados requieren un mayor esfuerzo e inversión a la hora de ser reparados o retirados para acceder al subsuelo.

\item Aspecto estético: actualmente los adoquines pueden diseñarse de todas formas, texturas, colores y disposiciones según las necesidades de la obra.
\end{itemize}

\section{Materias primas}
MANUAL EUROADOQUÍN
Las características que las materias primas deben cumplir, se contemplan en la futura norma Europea prEN 1338 (que Euroadoquín adopta)
\subsection{Cemento}
Cumplirá los requisitos establecidos en la norma UNE 80 301, los establecidos en la norma UNE 80 303 cuando se empleen cementos con características especiales y los establecidos en la
norma UNE 80 305 cuando se empleen cementos blancos.
\subsection{Áridos}
Se emplearán procedentes de río, de mina o piedras trituradas. La granulometría de los áridos que se utilicen será estudiada por el fabricante de manera que el producto terminado cumpla las características señaladas en la norma prEN 1338 (norma Europea).
\subsection{Arena}
\subsection{Agua}
Serán utilizadas, tanto para el amasado como para el curado, todas las aguas que no perjudiquen el fraguado y endurecimiento de los hormigones.
\subsection{Aditivos}
Se podrán utilizar adicciones y aditivos siempre que la sustancia agregada en las proporciones previstas, produzca el efecto deseado, sin perturbar las demás características del hormigón o mortero.
\subsection{Pigmentos}
Inorgánicos.
