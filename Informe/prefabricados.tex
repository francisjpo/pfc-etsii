%!TEX root = informe.tex
\chapter{Prefabricados del cemento. Adoquines}
\section{Descripción general}
A lo largo de la historia de la humanidad se han ido utilizando diferentes tipos de adoquines para pavimentar los suelos urbanos \cite{euroadoquin}. Los primeros adoquines eran de piedra, obtenidos a partir de los guijarros de río colocados sobre una capa de arena, usando una mezcla de cal y arena como sellante de juntas.

Debido al coste y el ruido del tráfico rodado, en la primera mitad del siglo XIX comenzaron a usarse los adoquines de madera, utilizando para el sellado residuos bituminosos. Debido a su reducida duración y a la posterior aparición de los neumáticos, los adoquines de madera son sustituidos por un modelo cerámico, con el que se usaba la misma arena tanto para la base como sellante.

Los adoquines de piedra seguían siendo más resistentes y además no eran tan deslizantes como los cerámicos, por lo que a finales del siglo XIX se comenzó la fabricación de los adoquines de hormigón. Estos proporcionaban una mayor uniformidad que los de piedra, eran muy resistentes y con un coste inferior. Alemania y Holanda fueron los primeros en incorporar este nuevo formato de adoquín a sus núcleos urbanos. Al principio se usaban modelos que imitaban a los de piedra tanto en forma como colocación, pero pronto se añadieron formas dentadas o curvas, permitiendo una mejor alineación con el trazado.

Finalmente, durante la década de los 70 se mejoraron sustancialmente los sistemas de fabricación, permitiendo una gran variedad de modelos de adoquines y un abaratamiento de los costes de fabricación e instalación.

\subsection{Ventajas del uso de adoquines}

En comparación con otros tipos de pavimentos tales como los asfálticos o los pavimentos contínuos hormigonados, los adoquines presentan las siguientes ventajas:

\begin{itemize}
\item Fabricación: no se utilizan derivados del petróleo, que suelen ser caros y contaminantes, además de requerir una mayor aportación de energía durante el proceso de fabricación. En contraposición, pueden utilizarse cementos y áridos locales, disminuyendo los costes de transporte.

El proceso de fabricación de los adoquines requiere una maquinaria específica debido a que son sometidos a presión y vibración para segurar una resistencia y durabilidad adecuadas. Esto implica un control sobre la fabricación, consistencia y fiabilidad del producto mayor que el resto de pavimentos.

\item Instalación: aunque los adoquines pueden colocarse de forma automatizada, están diseñados de base para ser colocados manualmente, permitiendo instalarse en zonas de difícil acceso, cargas elevadas (muelles de carga, aeropuertos, \ldots), resolver trazados complejos o pendientes pronunciadas. A diferencia de los pavimentos asfálticos, su ejecución no depende de la temperatura ambiente y pueden ser utilizados inmediatamente después de su finalización, lo que implica una reducción en los tiempos de ejecución de obra.

\item Comportamiento: los adoquines pueden ser diseñados para ser muy resistentes tanto a cargas verticales (distribuidas o puntuales) como a esfuerzos horizontales (aceleración-frenada, giros,\ldots). Además, soportan bien sin degradarse los vertidos de aceites y combustibles sobre el pavimento. Los niveles de ruido generados por el tráfico son similares o inferiores a otros pavimentos en ausencia de humedad y sensiblemente inferiores en condiciones de humedad, especialmente a bajas velocidades. La resistencia a deslizamiento es mayor al del resto de pavimentos.

\item Mantenimiento: la vida útil del adoquín viene determinada principalmente por el comportamiento de la base, subbase y explanada y no por el propio adoquín. La vida útil de cálculo suele ser a 30 años, aunque en condiciones normales puede superar los 50 años. De esta manera, al renovar el pavimento se pueden reutilizar entre un 90 y un 95\% de los adoquines originales \cite{euroadoquin}. El adoquín es la mejor opción en zonas donde aún no se han implantado todos los servicios de públicos debido a que pueden ser levantados fácilmente para llevar tareas de instalación o reparación en el subsuelo. La conservación de los adoquines se limita al relleno de juntas erosionadas con arena de sellado cada cierto tiempo y a la reposición de adoquines fracturados.

\item Costes: aunque inicialmente el precio del metro cuadrado instalado es algo superior a otros pavimentos, a largo plazo es mucho más barato debido al menor mantenimiento y la reutilización de piezas. Los pavimentos asfálticos y hormigonados requieren un mayor esfuerzo e inversión a la hora de ser reparados o retirados para acceder al subsuelo.

\item Aspecto estético: actualmente los adoquines pueden diseñarse de todas formas, texturas, colores y disposiciones según las necesidades de la obra.
\end{itemize}

\section{Materias primas}
Las características de las materias primas que se pueden emplear en la fabricación de los adoquines se contemplan en la norma UNE EN 1338:2004/AC:2006. En ella se especifican detalladamente los materiaes, propiedades, requisitos y métodos de ensayo de los adoquines prefabricados de hormigón no armados y accesorios complementarios, previstos para uso peatonal, uso en áreas sometidas a tráfico de vehículos y cubiertas, como por ejemplo: aceras, límites de áreas, sendas para bicicletas, aparcamientos, carreteras, autopistas, áreas industriales, aeropuertos, estaciones de autobuses y gasolineras. Esta norma no trata la visibilidad y la tactibilidad de los adoquines ni los adoquines permeables.

\subsection{Cemento}
El cemento es un conglomerante, formado a partir de arcilla y caliza (\ce{CaCO3}), que se endure al mezclarse con agua. Para producir cemento, la arcilla y la caliza se muelen juntas. A esta mezcla se le añade yeso para conferirle la propiedad de fraguar y endurecerse. El resultado se introduce en un horno rotatorio, normalmente seco por ser más eficiente energéticamente, a una temperatura aproximada de 1450\si{\celsius}. A continuación se introduce el material en un incinerador donde el calentamiento produce la liberación del \ce{CO2} de la caliza y se produce el cemento \emph{clinker}.

\begin{center}
\ce{CaCO3 + calor -> CaO + CO2}
\end{center}

El clinker es el óxido de calcio (\ce{CaO}) obtenido de la reacción anterior, que puede encontrarse acompañado de otros minerales como hierro, aluminio o silicio. El aporte de calor necesario para obtener el clinker representa la mayor parte del coste energético en la producción de cemento.

Por último, se produce la molienda del clinker junto yeso y otros materiales (bauxita, arena,\ldots) para mejorar sus propiedades, produciendo el cemento.

DIAGRAMA LUND UNIV. PAG 4 (Figure 1)

Cuando se utiliza la palabra \emph{cemento} se refiere normalmente a un cemento tipo Portland (supone un 95\% de la producción de cementos \cite{jsjunnesson}), nombre no comercial que implica un proceso de producción y una composición característicos. De acuerdo a la norma UNE-EN 197-1:2011 \cite{une1971}, el cemento se divide en tres grupos en función de la cantidad de cemento Portland incluido: CEM I, CEM II y CEM III. El CEM I (95\% a 100\% de contenido de cemento Portland) es el más usado en la fabricación de adoquines.

Con respecto a la normativa específica de adoquines, la norma UNE 80301:1996 \cite{une80301} en el ámbito de España establece los requisitos que debe tener el cemento común. Si se utilizan cementos especiales se recurrirá a la norma UNE 80303:2013, y si son blancos a la norma UNE 80305:2012.

\subsection{Áridos}
Los áridos (entre los que se incluye la arena) son particulas de roca puede ser tanto gravas (piedras de forma natural) como macadán (piedras trituradas), teniendo cada tipo una textura diferente. Se pueden añadir diferentes tamaños de áridos para ejercer una función diferente según el mismo. Fracciones menores rellenarán las cavidades que haya entre partículas mayores, aportando adherencia a costa de un mayor peso.

El material de machaqueo para la producción de macadán se criba para eliminar las partículas menores. Debido a que las partículas que forman el macadán son más irregulares, es más fiable como material de relleno por su capacidad de incrustamiento (la grava tiene una forma más redondeada). El macadán se puede ser utilizado de forma más generalizada en función de la localización ya que la grava natural es un recurso más limitado.

La fuente de recursos de áridos son principalmente de río, mina o cantera o piedras trituradas (macadán). La granulometría de los áridos que se utilicen deberá cumplir las características indicada en la norma UNE EN 1338:2004/AC:2006.

\subsection{Agua}
La calidad del agua es un factor muy importante a la hora de obtener un cemento de calidad y resistente. Por norma general se considera el agua potable como de suficiente buena calidad para usarse en la mezcla. También pueden usarse aguas de calidad inferior siempre que no perjudiquen el fraguado y endurecimiento del hormigón.

\subsection{Aditivos}
Se podrán utilizar adiciones o aditivos siempre que produzcan el efecto deseado (acelerante, retardante,\ldots) y no afecte a las características esperadas del hormigón.

\subsection{Reciclado}
Cuando las construcciones de cemento se demuelen, el cemento es normalmente reutilizables \cite{jsjunnesson}. El cemento se transporta a una estación de reciclaje donde es triturado hasta un tamaño adecuado con la nueva utilización que se le va a dar. Puede ser empleado como material de relleno para el pavimento de nuevas carreteras o como árido en la producción de más cemento. La reutilización del cemento conlleva una disminución del uso de recursos naturales tales como la piedra o la grava.

HABLAR DEL CASO CONCRETO DE ADOQUINES Y SU RECICLADO (95\% se vuelven a usar para pavimentar. 30 años.)
