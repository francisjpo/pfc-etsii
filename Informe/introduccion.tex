%!TEX root = informe.tex
\chapter{Introducción}
\section{Objeto}
El objetivo principal de este proyecto es el Análisis de Ciclo de Vida del adoquín común prefabricado del cemento utilizado en obras civiles y urbanismo. Se desea analizar el ciclo de vida del adoquín desde la obtención de la materia prima hasta su fin de vida.

Con este Análisis de Ciclo de Vida se pretende evaluar el comportamiento ambiental de las distintas etapas de su ciclo de vida, las cargas ambientales asociadas a estas etapas e identificar las posibles mejoras.

Para poder conseguir estos objetivos, se pueden establecer los siguientes objetivos básicos:
\begin{itemize}
\item Estudio y análisis de los procesos productivos, de instalación, mantenimiento y reciclado de los adoquines.
\item Análisis de las diferentes metodologías de Análisis de Ciclo de Vida.
\item Elección de una unidad de referencia del producto a partir de la cual se puedan normalizar los datos de entrada y salida y que permita su comparación con otros productos o con etapas de su ciclo de vida del mismo.
\item Desarrollo de un inventario de materiales y procesos de cada etapa del ciclo de vida.
\item Conocer y evaluar los impactos ambientales asociados a cada una de las etapas del producto y a su ciclo completo.
\end{itemize}

A su vez, la redacción del presente proyecto bajo la dirección del Departamento de Expresión Gráfica, Diseño y Proyectos de la Universidad de Málaga tiene como finalidad última la obtención del título de Ingeniero Industrial.

\section{Alcance}

En este proyecto se analiza la posibilidad de aprovechar la radiación solar, para la producción de energía eléctrica, en ubicaciones situadas en España y Rumanía, mediante centrales fotovoltaicas conectadas con la red eléctrica.

En el capítulo tres se presenta el consumo de energía eléctrica en el mundo. Para Es- paña se presenta también la evolución de la producción de energía eléctrica en función de su origen y se hace una previsión de la misma para el año 2008. Se hace una presentación de las energías renovables y de su nivel de implantación a nivel mundial, remarcando la energía solar fotovoltaica en España y Rumanía. A continuación, se presentan los principa- les componentes de una central fotovoltaica.

El capítulo cuatro muestra las fuentes de información utilizadas para la realización del proyecto.

En requisitos de diseño (capítulo 6) se presentan las dos ubicaciones inicialmente elegidas para el análisis: Tarragona y Bucarest.

El capitulo siete, análisis de soluciones, presenta los datos de radiación solar y tem- peratura para las dos ubicaciones elegidas inicialmente y también se buscan las ubicaciones con mayor radiación solar en España y Rumanía, presentando también en este caso la ra- diación solar y temperatura. Por último, se hace un análisis económico de las centrales fo- tovoltaicas emplazadas en cada ubicación para determinar el grado de rentabilidad de la inversión en cada caso.

Finalmente, el capítulo ocho presenta las mejores soluciones, tanto técnicas como económicas.

\section{Antecedentes}

La mayoría de las ciudades europeas utilizan materiales prefabricados basados en el cemento para urbanizar el terreno transformándolo en espacio público que utilizarán los ciudadanos. Estas instalaciones deben ser resistentes, económicas, funcionales y sobre todo sostenibles. La sostenibilidad es un requisito que ha ido ganando importancia en los últimos años debido no solo al aspecto económico —costes y mantenimiento principalmente— sino también al medioambiental.

El impacto medioambietal que producen las actividades humanas en la naturaleza debería convertirse en un elemento más de estudio en cualquier proyecto de ingeniería actual. Para el caso de este proyecto, el sector de las obras civiles y urbanismo supone un consumo muy elevado de materias primas y energía debido a que representa un porcentaje importante del Producto Interior Bruto de España, lo que implica altas emisiones al medio ambiente \cite{minetur}.

La incorporación de criterios ambientales en la fase de diseño de un producto y/o servicio se denomina \textit{ecodiseño}. El ecodiseño surge como respuesta a la necesidad de introducir estos criterios durante todo el ciclo de vida de un producto con el objetivo de prevenir o reducir su impacto ambiental —principalmente minimizar los residuos, emisiones y costes energéticos \cite{iso14006}. El ecodiseño es el eslabón clave hacia la sostenibilidad y el consumo responsable ya que incorpora nuevos conceptos como la visión de producto-sistema y el ciclo de vida e integra aspectos económicos y sociales como la ecoeficiencia y el ecodiseño sostenible \cite{ihobeeco}.

La metodología del Análisis de Ciclo de Vida (ACV) permite cuantificar todos los procesos relacionados con un producto y/o servicio desde el punto de vista de las \textit{entradas} —materias primas y energía— y \textit{salidas} —emisiones a la tierra, mar o aire y residuos— en el sistema, identificar los puntos clave y establecer una \textbf{estrategia de mejora} \cite{iso14040}.

\section{Metodología}

XXXX EXPLICAR MEJOR UNA VEZ TERMINADO EL PROYECTO XXXX
El presente proyecto seguirá el siguiente proceso de estudio:

\begin{itemize}
  \item En primer lugar se establecen las características del producto a estudio, historia de su desarrollo y materias primas que lo componen.
  \item Se expondrá la relevancia del producto con el impacto medioambiental y se explicará la metodología de Análisis de Ciclo de Vida, su importancia y las herramientas con las que se trabajará.
  \item A continuación se realizará un Análisis de Ciclo de Vida del producto desde ``la cuna hasta la tumba'', es decir: extracción de materias primas, fabricación, instalación, uso y mantenimiento, y fin de vida.
  \item Una vez analizados, se hará una comparación entre ellos.
  \item Por último, se desarrollarán las conclusiones obtenidas del estudio y posibles mejoras o futuras líneas de expansión.
\end{itemize}

\section{Normativa aplicada}

Para la realización de este proyecto se han tenido en cuenta la siguiente normativa:

\begin{itemize}
  \item UNE-EN-ISO 14040:2006, Gestión Ambiental. Análisis de ciclo de vida. Principios y marco de referencia.
  \item UNE-EN-ISO 14440:2006, Gestion Ambiental. Análisis de ciclo de vida. Requisitos y directrices.
  \item UNE-EN-ISO 150041EX:1998, Análisis de ciclo de vida simplificado.
  \item UNE-EN-ISO 14006:2011, Sistemas de gestión ambiental. Directrices para la incorporación del ecodiseño.
  \item UNE-EN 1338:2004/AC:2006 Adoquines de hormigón. Especificaciones y métodos de ensayo.
  \item UNE-EN 197-1:2011 La norma europea de especificaciones de cementos comunes.
  \item UNE 80301:1996 Cementos. Cementos comunes. Composicion, especificaciones y criterios de conformidad.
  \item UNE 127338:2007 Propiedades y condiciones de suministro y recepción de los adoquines de hormigón. Complemento nacional a la Norma UNE EN 1338.
  \item UNE-CEN/TR 15941:2011 IN Sostenibilidad en la construcción. Declaraciones ambientales de producto. Metodología para la selección y uso de datos genéricos.
  \item UNE-EN15804:2012 Sostenibilidad en la construcción.Declaracionesambientales de producto.
  \item UNE-EN 15978:2012 Sostenibilidad en la construcción. Evaluación del comportamiento ambiental de los edificios. Métodos de cálculo.
  \item ISO. UNE-ISO 21930 Sostenibilidad en la construcción de edificios. Declaración ambiental de productos de construcción.
\end{itemize}

\section{Bibliografía}

\section{Programas de cálculo}

Este proyecto ha sido tipografiado con el sistema de composición de documentos Xe\LaTeX. \LaTeX\ y Xe\LaTeX\ son distribuidos con el Mac\TeX, una redistribución de \TeX\ Live para OS X. La bibliografía ha sido generada mediante el sistema de gestión de referencias Bib\TeX.

Para la redacción se ha utilizado el editor de texto Sublime Text con el paquete LaTeXTools para la automatización de macros. Las fuentes de impresión empleadas son Minion Pro, Myriad Pro e Inconsolata-g.

Se ha utilizado un ordenador Apple Macbook Pro con sistema operativo OS X.

La generación de diagramas se ha realizado con MindNode para Mac y los paquetes de programación TikZ y PGF para Xe\LaTeX.

El software de Análisis de Ciclo de Vida elegido es SimaPro de PRé Consultants.

Los planos han sido creados usando el programa AutoCAD de Autodesk.

Los cálculos se han realizado mediante la aplicación de hoja de cálculo Numbers.app.

Las copias de seguridad se han realizado con el software de control de versiones Git en un repositorio online de Github y soporte local con Time Machine de Apple.

\section{Plan de gestión de la calidad aplicado durante la redacción del Proyecto}

Para la realización de este proyecto se ha aplicado la siguiente normativa:
\begin{itemize}
  \item UNE 157001:2002 Criterios generales para la elaboración de proyectos.
  \item UNE 50132:1994 Documentación. Numeración de las divisiones y subdivisiones en los documentos escritos.
  \item UNE 1027:1995 Dibujos técnicos. Plegado de planos.
  \item UNE 1032:1982 Dibujos técnicos. Principios generales de representación.
  \item UNE 1035:1995 Dibujos técnicos. Cuadro de rotulación.
  \item UNE 1039:1994 Dibujos técnicos. Acotación. Principios generales, definiciones, métodos de ejecución e indicaciones especiales.
\end{itemize}

Durante la redacción de este proyecto se han corroborado los datos aportados por el fabricante. Se han revisado errores de transcripción de datos, fallos en los cálculos, así como errores gramaticales y ortográficos. Además, se ha comprobado la consistencia de los conceptos de estudio con la metodología empleada. Por último, se ha utilizado un sistema de copia de seguridad basada en control de versiones, por la cual los datos pueden ser recuperados o consultados en cualquier momento.

\section{Definiciones y abreviaturas}
\begin{itemize}
  \item AENOR (Asociación Española de Normalización y Certificación): entidad de certificación de sistemas de gestión, productos y servicios, y responsable del desarrollo y difusión de las normas UNE.
  \item Análisis del Ciclo de Vida (ACV): recopilación y evaluación de las entradas, las salidas y los impactos ambientales potenciales de un sistema del producto a través de su ciclo de vida.
  \item Análisis del Inventario del Ciclo de Vida (ICV): fase del análisis del ciclo de vida que implica la recopilación y la cuantificación de entradas y salidas para un sistema del producto a través de su ciclo de vida.
  \item Aspecto ambiental: elemento de las actividades, productos o servicios de una organización que puede interactuar con el medio ambiente.
  \item BUWAL: Bundesamt für Unwelt, Wald und Landshaft. Oficina Federal de Medio Ambiente, Bosque y Campo (Suiza).
  \item Categoría de impacto: clase que representa asuntos ambientales de interés a la cual se pueden asignar los resultados del análisis del inventario del ciclo de vida.
  \item CEN: Comité Europeo de Normalización.
  \item Ciclo de vida: etapas consecutivas e interrelacionadas de un sistema del producto, desde la adquisición de materia prima o de su generación a partir de recursos naturales hasta la disposición final.
  \item De la cuna a la tumba: expresión que referencia al ciclo de vida de un producto desde la extracción de las materias primas hasta la disposición.
  \item Eco-Indicador99: indicador ambiental, desarrollado por PRé Consultants para el gobierno de Holanda.
  \item Ecodiseño: diseño que considera acciones orientadas a la mejora ambiental del producto o servicio en todas las etapas de su ciclo de vida, desde su creación en la etapa conceptual, hasta su tratamiento como residuo.
  \item Emisiones atmosféricas: introducción en la atmósfera por el hombre, de forma directa o indirecta, de sustancias o energía que tengan una acción perjudicial para la salud humana o el medio ambiente.
  \item Evaluación del Impacto del Ciclo de Vida (EICV): fase del análisis del ciclo de vida en la que los hallazgos del análisis del inventario o de la evaluación del impacto, o de ambos, se evalúan en relación con el objetivo.
  \item Impacto ambiental: alteración apreciable sobre la salud y bienestar de cualquier ser vivo o sobre el medio ambiente. En relación al ACV, se trata de la anticipación razonable a un efecto.
  \item ISO (International Standard Organitation). Organización Internacional de Estándares.
  \item Life Cycle Assessment (LCA): acrónimo en inglés de Análisis de Ciclo de Vida.
  \item Límite del sistema: conjunto de criterios que especifican cuales de los procesos unitarios son parte de un sistema del producto.
  \item Medio ambiente: conjunto de factores físico-químicos (agua, aire, clima, etc.), biológicos (fauna, flora y suelo) y socioculturales (asentamiento y actividad humana, uso y disfrute del territorio, formas de vida, etc.) que integran el entorno en que se desarrolla la vida del hombre y la sociedad (RD 4/1986 de 23 de enero 1986).
  \item Proceso unitario: elemento más pequeño considerado en el análisis del inventario del ciclo de vida para el cual se cuantifican datos de entrada y salida.
  \item UNE: Una Norma Española.
  \item UNE-EN: Una Norma Española que además es Norma Europea (European Norm) a través del CEN.
  \item UNE-EN-ISO: Adaptación de normativa ISO a ámbito europeo por el CEN y de ahí al ámbito español por AENOR.
  \item Unidad funcional: aquella prestación o función que realiza un producto que permite su comparación con otros.
  \item Vida útil: duración estimada que un objeto puede tener cumpliendo correctamente con la función para la cual ha sido creado.
\end{itemize}
