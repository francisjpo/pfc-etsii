\chapter{Introducción}
\section{Antecedentes}

La mayoría de las ciudades europeas utilizan materiales prefabricados basados en el cemento para urbanizar el terreno transformándolo en espacio público que utilizarán los ciudadanos. Estas instalaciones deben ser resistentes, económicas, funcionales y sobre todo sostenibles. La sostenibilidad es un requisito que ha ido ganando importancia en los últimos años debido no solo al aspecto económico — costes y mantenimiento principalmente — sino también al medioambiental.

El impacto medioambietal que producen las actividades humanas en la naturaleza ha pasado a ser un elemento más de estudio en cualquier proyecto de ingeniería actual. En el caso de este proyecto, el sector de las obras civiles y urbanismo supone un consumo muy elevado de materias primas y energía debido a que representa un porcentaje importante de la actividad económica de cualquier país occidental, lo que implica altas emisiones al medio ambiente.
De esta manera, utilizando la metodología de Análisis de Ciclo de Vida (ACV) se pretende conocer con una rigurosidad adecuada el ciclo de vida de un producto y/o servicio, evaluando el impacto potencial sobre el medio ambiente a lo largo de su vida.

\section{Objetivos y alcance}
El objetivo principal de este proyecto es el Análisis de Ciclo de Vida de un adoquín común utilizado en obras civiles y urbanismo. Se pretende analizar todas las entradas y salidas tanto de materiales como de energía desde la extracción de la materia prima hasta el final de vida del producto, además de identificar y clasificar los principales aspectos medioambientales y sus correspondientes impactos en los diferentes procesos que intervienen en su fabricación. De esta forma, se pueden establecer los siguientes objetivos básicos:
\begin{itemize}
\item análisis del ciclo de vida de las materias primas hasta que llegan a la planta de fabricación.
\item análisis del ciclo de vida de los procesos productivos involucrados en el proceso de fabricación hasta su salida.
\item análisis del ciclo de vida del producto hasta su final de vida.
\end{itemize}

A su vez, la redacción del presente proyecto bajo la dirección del Departamento de Expresión Gráfica, Diseño y Proyectos de la Universidad de Málaga tiene como finalidad última la obtención del título de Ingeniero Industrial.
